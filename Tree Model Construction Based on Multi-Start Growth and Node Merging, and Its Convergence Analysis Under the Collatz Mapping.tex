%===================== Tree Model Paper Template =====================
\documentclass[12pt]{article}

%------------ Packages ------------
\usepackage{amsmath,amsthm,amssymb}
\usepackage{geometry}
\usepackage{graphicx}
\usepackage{hyperref}
\usepackage{longtable}
\usepackage{booktabs}
\usepackage{cite}
\usepackage{enumitem}
\usepackage{float}
\usepackage{amsthm}
\usepackage{tikz}
\usetikzlibrary{arrows.meta, shapes, positioning, calc, trees} 

% 用于定义定理类环境

%------------ Page settings ------------
\geometry{margin=1in}

%------------ Theorem environments ------------
\newtheorem{definition}{Definition}[section]
\theoremstyle{plain}
\newtheorem{theorem}{Theorem}[section]
\newtheorem{lemma}[theorem]{Lemma}
\newtheorem{proposition}[theorem]{Proposition}
\newtheorem{corollary}[theorem]{Corollary}

\newtheorem{example}[theorem]{Example}

\theoremstyle{remark}
\newtheorem{remark}[theorem]{Remark}
\title{Tree Model Construction Based on Multi-Start Growth and Node Merging, and Its Convergence Analysis Under the Collatz Mapping}
\author{Jiang Yuxiao}
\date{\today}

\begin{document}
	\maketitle
	
	\begin{abstract}
		This paper proposes a tree-model construction method based on multi-start growth processes, branching extensions, and node-merging rules. Starting from infinitely many initial points that satisfy prescribed constraints, the model generates branch nodes through a defined growth function and branching rule, forming a branch-node set for each starting point. Each starting point together with its branch-node set constitutes an extension path. When the result of a branching process coincides with an existing starting point, the connection rule automatically merges the corresponding structures, thereby forming a globally ordered, acyclic, and convergent tree-like structure. The resulting structure satisfies five fundamental properties: acyclicity, convergence, node coverage, node uniqueness, and uniqueness of parent nodes. Based on these properties, the paper proves that the constructed model is, in the strict sense, an arborescence (a rooted directed tree), and further verifies its convergence completeness under the Collatz mapping. The model provides a complete structural description of the global convergence behavior of the \(3n+1\) transformation and offers a self-organizing and verifiable unified framework for structured proofs in discrete dynamical systems.
	\end{abstract}
	\tableofcontents
	\newpage
	\section{Introduction}
	
	Since its proposal in 1937, the Collatz conjecture (also known as the $3n+1$ conjecture) has centered on the claim that for any positive integer $N$, the iterative mapping
	\[
	T(N)=
	\begin{cases}
		N/2, & \text{if $N$ is even},\\[4pt]
		3N+1, & \text{if $N$ is odd},
	\end{cases}
	\]
	must reach $1$ after a finite number of steps. Despite its extraordinarily simple formulation, the conjecture has resisted all known mathematical techniques for more than eighty years, becoming one of the most renowned open problems of modern mathematics. Traditional research has largely focused on the statistical properties of iteration trajectories, dynamical-system characteristics, and computational complexity, yet lacks a unified modeling framework capable of describing all positive integers while simultaneously exhibiting strong algebraic closure.
	This work introduces a new perspective: the construction of a directed tree model that covers all positive integers (hereafter referred to as the “complete tree model”). Its odd-number substructure (the “odd tree”) is composed of three key components:
	
	\begin{itemize}
		\item \textbf{Starting set}: all positive odd integers are partitioned into three classes according to their residue modulo $6$, and from each odd starting point its growth sequence is generated;
		\item \textbf{Growth-node set}: for each odd starting point, repeated application of multiplication by $2$ produces all of its even descendants;
		\item \textbf{Branch-node set}: each growth node is examined to determine whether the reversible branching operation $(v-1)/3$ yields an integer, thereby providing connection nodes for other starting points.
	\end{itemize}
	
	Through a connection rule, the structures derived from different starting points are merged, ultimately forming a directed tree rooted at $1$. This tree possesses the following properties:
	
	\begin{enumerate}
		\item Its odd-number portion covers all positive odd integers and satisfies unique parenthood, acyclicity (except for a special self-loop), and consistency of direction;
		\item After the inclusion of even nodes, the resulting complete tree covers all positive integers while preserving all aforementioned tree properties;
		\item Every node in the tree can return to the root node $1$ through a finite number of reverse steps.
	\end{enumerate}
	
	The reverse regression process in the above tree corresponds exactly to the descent process of the Collatz mapping: reverse growth of even nodes corresponds to \(N \mapsto N/2\), and reverse branching of odd nodes corresponds to \(N \mapsto 3N - 1\) (the operation by which a Collatz odd term maps to an even term). Therefore, proving that every node in the tree model eventually regresses to \(1\) is equivalent to proving that Collatz iteration must necessarily descend to \(1\).
	The main contributions of this study can be summarized as follows:
	
	\begin{enumerate}
		\item A complete algebraic tree model covering all positive integers is constructed, with rigorous proofs of node uniqueness, edge uniqueness, acyclicity, and uniqueness of parent nodes;
		\item A generating function for the branch-node set is provided, and its completeness and uniqueness for all odd starting points are rigorously established;
		\item Using a potential-function approach, the reverse-regression property of any node in the tree is proven, forming the key step in demonstrating the descent property under the Collatz mapping;
		\item The exact bidirectional correspondence between the Collatz mapping and the tree model’s reverse regression is established;
		\item Based on all the above structures, a “Completeness-Closure Main Theorem” is proposed, which asserts that if the growth and branching systems satisfy completeness and uniqueness, no node in the complete tree can escape the root node $1$, thereby equivalently proving that all orbits of $T$ must eventually terminate at $1$.
	\end{enumerate}
	
	The structure of this paper is as follows: Chapter 2 provides the basic definitions; Chapter 3 presents the starting set and growth rules; Chapter 4 introduces the branching conditions and generating functions; Chapter 5 describes the connection rules; Chapter 6 constructs the odd-tree model; Chapter 7 presents the main theorem of the odd tree with rigorous proof; Chapter 8 discusses potential functions and reverse regression; Chapter 9 extends the model to the complete tree; Chapter 10 establishes the bidirectional correspondence between the Collatz mapping and the tree model; Chapter 11 presents the Completeness-Closure Main Theorem; Chapter 12 discusses possible counterarguments; Chapter 13 concludes the paper; and the appendix provides exhaustive verification of short cycles, cross-layer conflicts, and a formal definition of the graph structure.
	
	The goal of this study is not to propose a “new technique” for the Collatz conjecture in the traditional sense, but to provide a fully structured algebraic and graph-theoretical framework in which each generative step can be formally verified, thereby supplying a rigorous logical foundation for the ultimate global descent property.
	\section{Basic Definitions and Preliminary Knowledge}
	
	This section provides the fundamental definitions required for the proof, including the notation system, integer decomposition, the standard form of the Collatz mapping, and several preliminary lemmas used in the subsequent tree-model construction.
	
	\subsection{Notation and Set Conventions}
	
	The following conventions are adopted throughout this paper:
	
	\begin{itemize}
		\item $\mathbb{N} = \{1,2,3,\dots\}$ denotes the set of positive integers.
		\item $\mathbb{N}_{\mathrm{odd}} = \{\, n \in \mathbb{N} \mid n \text{ is odd} \,\}$.
		\item $\mathbb{N}_{\mathrm{even}} = \{\, n \in \mathbb{N} \mid n \text{ is even} \,\}$.
		\item For any $m \in \mathbb{N}$, $m$ can be uniquely expressed as $m = 2^k t$, where $t$ is odd and $k \ge 0$. The integer $k$ is called the \emph{2-adic valuation} of $m$, denoted by $\nu_2(m) = k$.
	\end{itemize}
	
	Unless otherwise specified, all variables $n,x,k$ in this paper denote positive integers.
	
	\subsection{Standard Form of the Collatz Mapping}
	
	The Collatz iteration is defined as a mapping $T:\mathbb{N} \to \mathbb{N}$:
	\[
	T(n) =
	\begin{cases}
		n/2, & n \equiv 0 \pmod{2},\\[2mm]
		3n+1, & n \equiv 1 \pmod{2}.
	\end{cases}
	\]
	
	For any $n \in \mathbb{N}$, its orbit is defined as
	\[
	\mathcal{O}(n) = \{\, T^k(n) \mid k \ge 0 \,\}.
	\]
	
	The classical Collatz conjecture asserts that for all positive integers $n$, the orbit $\mathcal{O}(n)$ eventually enters the cycle $(4,2,1)$.
	\subsection{Parity Decomposition and the ``Descent–Ascent'' Structure}
	
	Since even steps can occur consecutively, the Collatz orbit can be normalized as
	\[
	n \xrightarrow{\times 3+1} 3n+1 \xrightarrow{\div 2^{\nu_2(3n+1)}} \text{odd number}.
	\]
	
	This provides a simplified description of the ``ascent–descent'' process: an odd number is first elevated by the $3n+1$ operation and then consecutively divided by 2 until the next odd number is reached.
	
	\begin{definition}[Odd Regression Mapping]
		Define the \emph{regression mapping} $R:\mathbb{N}_{\mathrm{odd}} \to \mathbb{N}_{\mathrm{odd}}$ between odd numbers as
		\[
		R(m) = \frac{3m+1}{2^{\nu_2(3m+1)}}.
		\]
	\end{definition}
	
	The iteration of $R$ is equivalent to the Collatz iteration restricted and folded onto the odd-number layer.
	
	\subsection{Preliminary Results on $6n \pm 1$}
	
	To construct the tree model starting from odd numbers, the following basic facts are needed.
	
	\begin{lemma}[Modulo-6 Classification of Odd Numbers]
		Every odd number belongs to exactly one of the following three classes:
		\[
		\mathbb{N}_{\mathrm{odd}}
		= \{\, 6n-5, \; 6n-3, \; 6n-1 \mid n \in \mathbb{N} \,\}.
		\]
	\end{lemma}
	
	\begin{lemma}[2-adic Valuation of $3m+1$]
		For an odd number $m$, the 2-adic valuation $\nu_2(3m+1)$ depends solely on the residue class of $m \bmod 6$.
	\end{lemma}
	
	The proof is provided in Appendix C in the section on the formal description of the graph structure. This fact ensures logical consistency when constructing growth nodes and branch nodes.
	
	\subsection{Necessity of the Tree Model}
	
	When studying the Collatz mapping from a reverse perspective, one must clarify \emph{which types of numbers may serve as branching sources} (that is, those satisfying the divisibility condition $(\cdot - 1)/3$).  
	In the forward Collatz rule, an odd number $u$ is mapped to
	\[
	3u+1,
	\]
	which is always even; subsequently, this even number is repeatedly divided by $2$ until the next odd number appears. Therefore, in the reverse construction we should focus on these intermediate \emph{even} values, rather than on arbitrary integers.
	
	We make the following observations:
	
	\begin{itemize}
		\item Let an intermediate even number be denoted by \(m\). If there exists an odd number \(u\) such that \(m = 3u + 1\), then \(m\) must be even, and we can verify that
		\[
		\frac{m - 1}{3} = u \in \mathbb{N}.
		\]
		Thus, the operation “divide by \(3\)” is, in essence, intended to act on those even numbers \(m\) that arise from applying \(3\cdot +1\) to some odd input.
		
		\item In the reverse construction, an even number \(m\) typically appears within the growth chain of some odd starting point \(s\) (that is, in the sequence \(s \mapsto 2s \mapsto 4s \mapsto \cdots\)). Therefore, one must examine whether a growth-chain even number \(m = 2^k s\) satisfies $(m - 1)/3 \in \mathbb{N}$ and whether the result is odd; if so, that result becomes a \emph{branch node}.
		
		\item To remain consistent with the convention that the ``odd tree'' contains only odd nodes, we distinguish two sets:
		\begin{itemize}
			\item \emph{Growth-node set (even numbers)}:  
			For each odd starting point \(s\), consider its even growth chain $\{2^k s\}_{k\ge 1}$. These even numbers do not appear as nodes of the odd tree but play an essential intermediary role in the branching construction.
			
			\item \emph{Branch-node set (odd numbers)}:  
			If some even number \(m = 2^k s\) in the growth chain satisfies $(m - 1)/3 \in \mathbb{N}$ and the quotient is odd, then this odd number is included in the branch-node set and becomes a node of the odd tree.
		\end{itemize}
		
		\item Consequently, when determining whether an odd number \(v\) is a branch node of another starting point, the correct reverse test is:
		\[
		\text{Does there exist a starting point } s \text{ and } k \ge 0 \text{ such that }
		m = 2^k s \text{ and } \frac{m - 1}{3} = v \in \mathbb{N}_{\mathrm{odd}}?
		\]
		In other words: one must first locate an even number \(m\) in some growth chain and then test $(m - 1)/3$.
	\end{itemize}
	
	In summary, the expression $(\cdot - 1)/3$ in reverse analysis should act on \emph{intermediate even numbers} (the outcomes of the forward map $3u+1$), rather than on arbitrary integers. Although the odd tree contains only odd nodes, determining branching relations necessarily requires the involvement of even numbers on the growth chains as intermediaries. The odd numbers produced via $(m - 1)/3$ from these even numbers form the branch-node set and become nodes of the odd tree.
	
	In the following sections, these structures constitute the core framework of the odd-tree model and are used to establish completeness and acyclicity of the odd nodes.
	\section{Starting Set and Growth Rules}
	\label{sec:starting-growth}
	
	This section presents the first step in constructing the positive-integer tree model: the definition of the starting set and the growth-node set generated from the starting points. This part forms the foundational structure of the entire tree model and determines the classification of odd nodes, the uniqueness of growth paths, and the generation of subsequent branch nodes.
	
	\subsection{Definition of the Starting Set}
	
	We take all positive odd integers as the starting set of the tree model and partition them into three classes according to their residue modulo $6$.  
	
	Let
	\[
	\mathbb{O}^{+} = \{\, 2k-1 \mid k \in \mathbb{N} \,\}
	\]
	denote the set of all positive odd integers. We then define:
	
	\begin{definition}[Starting Set]
		\label{def:starting-set}
		The starting set $\mathcal{S}$ of the tree model is defined as
		\[
		\mathcal{S} = \{\, 6n-5, \; 6n-3, \; 6n-1 \mid n \in \mathbb{N} \,\}.
		\]
		The three subclasses are denoted by
		\[
		\mathcal{S}_{1} = \{\, 6n-5 \mid n \in \mathbb{N} \,\}, \quad
		\mathcal{S}_{3} = \{\, 6n-3 \mid n \in \mathbb{N} \,\}, \quad
		\mathcal{S}_{5} = \{\, 6n-1 \mid n \in \mathbb{N} \,\}.
		\]
		Clearly, $\mathcal{S} = \mathbb{O}^{+}$, and the three classes are mutually disjoint and collectively cover all positive odd integers.
	\end{definition}
	
	The mathematical role of this classification is as follows:
	
	\begin{itemize}
		\item The classification aligns with the reverse condition of the $(3n+1)$ operation;
		\item Only $\mathcal{S}_{1}$ and $\mathcal{S}_{5}$ produce branch nodes;
		\item $\mathcal{S}_{3}$ does not produce branch nodes, naturally corresponding to the subsequent uniqueness proofs;
		\item The reverse-path structure of each odd starting point (i.e., the structure in this tree model) is uniquely determined by its residue modulo $6$.
	\end{itemize}
	\subsection{Definition of Growth Rules}
	
	Starting from any initial point $s \in \mathcal{S}$, all of its growth nodes can be obtained by repeatedly multiplying by $2$:
	
	\begin{definition}[Growth-Node Set]
		\label{def:growth-set}
		For any starting point $s \in \mathcal{S}$, its growth-node set is defined as
		\[
		\mathcal{G}(s) = \{\, s \cdot 2^{k} \mid k \in \mathbb{N}_0 \,\}.
		\]
		Here, $k = 0$ corresponds to the starting point itself, and $k \ge 1$ corresponds to the successive layers of even nodes.
	\end{definition}
	
	Thus, from each starting point, a unique growth chain can be constructed upwards (or to the right):
	\[
	s \longrightarrow 2s \longrightarrow 4s \longrightarrow 8s \longrightarrow \cdots
	\]
	
	Under this definition of growth, for any even node $y \in \mathcal{G}(s)$, one can return to the starting point $s$ through $k$ successive divisions by 2.
	
	\subsection{Basic Properties of Growth Nodes}
	
	The growth-node set satisfies the following properties:
	
	\begin{lemma}[Uniqueness of Growth Nodes]
		\label{lem:growth-unique}
		For any $s \in \mathcal{S}$, each node in $\mathcal{G}(s)$ has a unique predecessor, which is uniquely determined by the division-by-2 operation.
	\end{lemma}
	
	\begin{lemma}[Non-Intersection Property]
		\label{lem:growth-no-intersection}
		For any distinct $s_1, s_2 \in \mathcal{S}$ with $s_1 \ne s_2$, we have
		\[
		\mathcal{G}(s_1) \cap \mathcal{G}(s_2) = \varnothing.
		\]
	\end{lemma}
	
	The proof relies on the invariance of the highest odd part in the binary representation: the least significant odd portion of different starting points is distinct, so their binary expansion chains cannot intersect.
	\subsection{Illustrative Example of the Structure}
	
	The following shows a growth example for the starting point $s = 6n-5 \in \mathcal{S}_1$ (only a partial set of nodes is displayed):
	\[
	6n-5 \;\longrightarrow\; 2(6n-5) \;\longrightarrow\; 4(6n-5) \;\longrightarrow\; 8(6n-5) \;\longrightarrow \cdots
	\]
	
	For instance, taking $s = 13$ (corresponding to $n = 3$), the growth chain is
	\[
	13 \;\longrightarrow\; 26 \;\longrightarrow\; 52 \;\longrightarrow\; 104 \;\longrightarrow\; 208 \;\longrightarrow \cdots
	\]
	
	This growth chain will be combined with the branching rules in the next section to construct the complete tree structure.
	\section{Branching Conditions and Branch-Node Set}
	\label{sec:branch-set}
	
	This section defines the branch nodes generated from growth nodes and provides the complete generating function for branch nodes. Branch nodes serve to connect structures derived from different starting points, ultimately merging them into a unified directed tree.
	
	\subsection{Definition of Branching Conditions}
	
	Starting from any initial point $s \in \mathcal{S}$, its growth-node set is
	\[
	\mathcal{G}(s) = \{\, s \cdot 2^k \mid k \in \mathbb{N}_0 \,\}.
	\]
	
	If a growth node $y \in \mathcal{G}(s)$ satisfies
	\[
	\frac{y-1}{3} \in \mathbb{N},
	\]
	then $y$ is said to satisfy the branching condition, and a new odd node is defined as follows:
	
	\begin{definition}[Branch Node]
		\label{def:branch-node}
		Let $y \in \mathcal{G}(s)$. If $(y-1)/3$ is a positive integer, define
		\[
		b = \frac{y-1}{3},
		\]
		and call $b$ the branch node of $y$.
	\end{definition}
	
	This condition corresponds exactly to the odd-step inverse of the Collatz mapping:
	\[
	3b + 1 = y,
	\]
	so every growth node satisfying the condition generates an outward-extending ``reverse odd branch.''
	
	Under this branching condition, for any odd $b \in \mathcal{B}(s)$, after applying the odd step of the Collatz mapping, one can return to the starting point $s$ through $k$ successive divisions by 2. In other words, for any odd $b \in \mathcal{B}(s)$, the Collatz iteration necessarily reaches $s$, with the descending even portion of the process included within this odd-step transformation.
	
	\subsection{Branch-Node Set}
	
	\begin{definition}[Branch-Node Set]
		\label{def:branch-set}
		For any starting point $s \in \mathcal{S}$, the set of all corresponding branch nodes forms the branch-node set:
		\[
		\mathcal{B}(s) = 
		\left\{\, \frac{s \cdot 2^k - 1}{3} \;\middle|\; k \in \mathbb{N}_0, \; s \cdot 2^k \equiv 1 \pmod{3} \right\}.
		\]
	\end{definition}
	
	Note that since $2 \equiv -1 \pmod{3}$, we have
	\[
	s \cdot 2^k \equiv s(-1)^k \pmod{3},
	\]
	so the parity of $k$ that produces branches differs depending on the modulo-6 class of the starting point.
	\subsection{Branching Behavior of the Three Types of Starting Points}
	
	Based on the classification of $s \bmod 6$, we have:
	
	\[
	\begin{array}{c|c|c}
		\text{Starting-Point Type} & s \bmod 6 & \text{Generates Branch?} \\
		\hline
		\mathcal{S}_1 & 1 & \text{Yes (only for even $k$)} \\
		\mathcal{S}_3 & 3 & \text{Never} \\
		\mathcal{S}_5 & 5 & \text{Yes (only for odd $k$)}
	\end{array}
	\]
	
	Formally:
	
	\begin{itemize}
		\item If $s = 6n-5 \in \mathcal{S}_1$, then
		\[
		s \cdot 2^k \equiv 1 \cdot (-1)^k \pmod{3},
		\]
		so $k$ must be even.
		
		\item If $s = 6n-3 \in \mathcal{S}_3$, then
		\[
		s \cdot 2^k \equiv 0 \pmod{3},
		\]
		so no branches are ever generated.
		
		\item If $s = 6n-1 \in \mathcal{S}_5$, then
		\[
		s \cdot 2^k \equiv 2(-1)^k \equiv -1 \cdot (-1)^k \pmod{3},
		\]
		so $k$ must be odd.
	\end{itemize}
	
	\subsection{Branch-Node Generating Functions}
	
	The explicit generating functions for starting points in $\mathcal{S}_1$ and $\mathcal{S}_5$ are as follows.
	
	\subsubsection{(1) Branch-Node Generating Function for $\mathcal{S}_1$: $s = 6n-5$}
	
	Since branches occur only when $k = 2x$, we have
	\[
	b = \frac{(6n-5) \cdot 2^{2x} - 1}{3}, \qquad x \in \mathbb{N}_0.
	\]
	
	This can be simplified to
	\[
	b = (8n-7) \cdot 4^{x-1} + \frac{4^{x-1}-1}{3}, \quad x \ge 1.
	\]
	
	\begin{definition}[Branch-Node Generating Function for $\mathcal{S}_1$]
		\label{def:branch-s1}
		\[
		B_1(n,x) = \frac{(6n-5) \cdot 2^{2x}-1}{3} \equiv (8n-7) \cdot 4^{x-1} + \frac{4^{x-1}-1}{3}.
		\]
	\end{definition}
	
	\subsubsection{(2) Branch-Node Generating Function for $\mathcal{S}_5$: $s = 6n-1$}
	
	Branches occur only when $k = 2x-1$ (odd), giving
	\[
	b = \frac{(6n-1) \cdot 2^{2x-1} - 1}{3}.
	\]
	
	This can be rewritten as
	\[
	b = (4n-1) \cdot 4^{x-1} + \frac{4^{x-1}-1}{3}, \quad x \ge 1.
	\]
	
	\begin{definition}[Branch-Node Generating Function for $\mathcal{S}_5$]
		\label{def:branch-s5}
		\[
		B_5(n,x) = \frac{(6n-1) \cdot 2^{2x-1} - 1}{3} \equiv (4n-1) \cdot 4^{x-1} + \frac{4^{x-1}-1}{3}.
		\]
	\end{definition}
	
	\subsubsection{(3) $\mathcal{S}_3$: $s = 6n-3$ Generates No Branches}
	
	Since $6n-3 \equiv 0 \pmod{3}$, all growth nodes satisfy
	\[
	y \equiv 0 \cdot (-1)^k \equiv 0 \pmod{3},
	\]
	and thus never meet the branching condition $y \equiv 1 \pmod{3}$.
	
	\[
	\mathcal{B}(6n-3) = \varnothing.
	\]
	\subsection{Example}
	
	For instance, take $s = 13 = 6 \cdot 3 - 5 \in \mathcal{S}_1$. For $x = 1$, we have:
	\[
	B_1(3,1) = \frac{(6 \cdot 3 - 5) \cdot 4 - 1}{3} 
	= \frac{13 \cdot 4 - 1}{3} 
	= \frac{51}{3} = 17.
	\]
	
	Similarly, take $s = 17 = 6 \cdot 3 - 1 \in \mathcal{S}_5$. For $x = 1$, we have:
	\[
	B_5(3,1) = \frac{(6 \cdot 3 - 1) \cdot 2 - 1}{3} 
	= \frac{17 \cdot 2 - 1}{3} 
	= \frac{33}{3} = 11.
	\]
	
	These nodes will serve as key connecting points in the subsequent connection rules.
	\section{Connection Rules}
	\label{sec:connection-rule}
	
	In the previous section, we defined the growth node set $\mathcal{G}(s)$ and the branch node set $\mathcal{B}(s)$ generated from the starting set $\mathcal{S}$. This section presents the core mechanism for merging structures generated by different starting points into a single directed tree—the \emph{connection rules}.
	
	The purpose of the connection rules is as follows:  
	\emph{When a branch node generated by a growth node from one starting point exactly coincides with another starting point, the two previously independent structures should be merged at this node, ensuring that all odd nodes ultimately form a single unified tree structure.}
	
	\subsection{Formal Definition of the Connection Rule}
	
	\begin{definition}[Connection Rule]
		\label{def:connection}
		Let $s_1, s_2 \in \mathcal{S}$ be two starting points.  
		If there exists a growth node $y \in \mathcal{G}(s_1)$ satisfying the branch condition and generating a branch node
		\[
		b = \frac{y-1}{3} \in \mathcal{B}(s_1),
		\]
		and this branch node coincides exactly with another starting point:
		\[
		b = s_2,
		\]
		then in the tree structure, connect the node $b$ to its parent $y$ as follows:
		\[
		y \longrightarrow b = s_2.
		\]
		
		Furthermore, merge the entire structure generated by $s_1$ 
		(all its growth chains and branch node chains) with the entire structure generated by $s_2$, making them adjacent parts of the same tree. This process is referred to as \emph{structure fusion}.
	\end{definition}
	
	Note:  
	- This rule is triggered only when a branch node exactly “collides” with another starting point;  
	- After fusion, $s_2$ is no longer an independent root but becomes a non-root node within the overall tree.
	
	The connection rule is the key mechanism by which the odd-numbered tree ultimately becomes a single-rooted tree (with root $1$).
	
	\subsection{Special Case $b=1$}
	
	When the branch node $b$ equals $1$, it indicates that this branch node comes from the unique growth node capable of generating $1$:
	\[
	y = 4.
	\]
	
	Since $4$ can only be generated by the growth nodes of starting point $1$ ($1 \cdot 2^2 = 4$), we have:
	\[
	4 \longrightarrow 1
	\Rightarrow
	1 \longrightarrow 4 \longrightarrow 1
	\]
	
	This forms a special self-loop of length $2$ (or length $3$ if including $2$):
	\[
	1 \to 4 \to 2 \to 1.
	\]
	
	This loop is the only exception allowed in the construction and does not affect the single-rooted property of the tree:
	
	- $1$ is always regarded as the root of the entire odd-numbered tree;  
	- The loop occurs only in the even-numbered part ($2,4$);  
	- The odd-numbered structure remains acyclic.
	\subsection{Example}
	
	Consider two starting points $s_1 = 13 \in \mathcal{S}_1$ and $s_2 = 17 \in \mathcal{S}_5$.
	
	According to the generation functions from the previous section:
	
	- The growth node $y = 52$ of $13$ (i.e., $x=1$) generates a branch node
	\[
	b = \frac{52-1}{3} = 17.
	\]
	
	Thus, we have
	\[
	b = s_2,
	\]
	which triggers the connection rule and produces the edge
	\[
	52 \to 17.
	\]
	
	Consequently, the structure generated by $13$ merges with the structure of $17$ at node $17$.
	
	\subsection{Role of the Connection Rule}
	
	The connection rule ensures that:
	
	1. **All odd nodes ultimately merge into a single connected structure;**  
	2. **Each node has a unique parent (branch nodes can only originate from a unique growth chain);**  
	3. **The tree ultimately has only one true root: $1$;**  
	4. **Branch node sequences from different starting points do not conflict or overlap, ensuring that the entire tree is free of multiple values and acyclic (except for the special $1\!-\!2\!-\!4$ loop).**
	
	The connection rule serves as the core \emph{gluing mechanism} for both the odd-numbered tree and the complete tree of positive integers, and forms the foundation for the subsequent main theorem (tree properties) and the correspondence with the Collatz mapping.
	\section{Formal Construction of the Odd-Numbered Tree Model}
	
	In this section, based on the previously defined starting set, branch conditions, and connection rules, we provide a rigorous construction of the \emph{odd-numbered Collatz tree model}. This structure contains only odd nodes; all nodes are positive odd integers, and all edges arise from branch generation rules and connection rules. The goal is to construct a directed tree in the odd domain where each node has a unique parent, the structure is acyclic, and the root is uniquely $1$, thereby laying the foundation for subsequent discussions on convergence, potential function descent, and completeness.
	
	\subsection{Node Set}
	
	The starting set is defined as
	\[
	\mathcal{S} = \{6n-5,\, 6n-3,\, 6n-1 \mid n \in \mathbb{N}\},
	\]
	where each element is a positive odd number. For any starting point $s \in \mathcal{S}$, its branch node set can be defined according to the branch generation functions as
	\[
	B_s = \{\, b_{s,x} \mid x \ge 1 \,\},
	\]
	where each $b_{s,x}$ is an odd number.
	
	The node set of the odd-numbered tree consists of all starting points and their corresponding branch nodes:
	\[
	\mathcal{V}_{\mathrm{odd}} = \mathcal{S} \ \cup \ \bigcup_{s \in \mathcal{S}} B_s.
	\]
	Hence, $\mathcal{V}_{\mathrm{odd}}$ contains only odd nodes and no even growth-chain nodes.
	
	\subsection{Edges of the Odd-Numbered Tree}
	
	The directed edges in the odd-numbered tree are divided into two types: \emph{branch edges} and \emph{merging edges}.
	
	\paragraph{(1) Branch Edges}
	
	For any starting point $s \in \mathcal{S}$ and any branch node $b_{s,x} \in B_s$, define a directed edge
	\[
	s \longrightarrow b_{s,x},
	\]
	representing the direct odd successor generated from the starting point $s$ via the branch generation function.
	
	\paragraph{(2) Merging Edges}
	
	If a branch node $b_{s,x}$ coincides with another starting point $s' \in \mathcal{S}$, i.e.,
	\[
	b_{s,x} = s',
	\]
	then in the global structure, this node is merged as a single node. The node $b_{s,x}$ retains both:
	\[
	s \longrightarrow b_{s,x},
	\qquad
	b_{s,x} = s' \longrightarrow b_{s',y} \quad (y \ge 1).
	\]
	
	That is, the node inherits its incoming edge as a branch node and all outgoing edges as a starting point. Consequently, a node may have multiple children, but its parent is uniquely determined by the construction.
	
	More generally, if a branch node $b$ equals an existing odd node (either starting or branch node), it is treated as the same node and directly merged, preventing duplicate nodes.
	
	\begin{example}[Merging Example]
		Consider starting points $1$ and $5$, whose branch chains are
		\[
		1 \to 5 \to 21 \to 85 \to \cdots,
		\qquad
		5 \to 3 \to 13 \to 53 \to \cdots.
		\]
		Since $5$ is both a branch node of $1$ and an independent starting point, the structures merge at node $5$. Node $5$ now has two outgoing edges:
		\[
		5 \longrightarrow 21,
		\qquad
		5 \longrightarrow 3.
		\]
		Here, $21 \equiv 3 \pmod{6}$ has an empty branch node set and generates no successors; if $85$ coincides with another starting point, it merges according to the same rules.
	\end{example}
	
	\paragraph{Special Case: Branch Node $1$}
	
	The only possible branch node $1$ comes from the even number $4$, which originates solely from the growth chain of starting point $1$. Thus, the self-loop
	\[
	1 \to 2 \to 4 \to 1
	\]
	is invisible in the odd-numbered tree and does not affect the acyclicity of the odd structure.
	\subsection{Definition of the Odd-Numbered Tree Model}
	
	Combining the above rules for nodes and edges, the odd-numbered tree model is defined as a directed graph
	\[
	\mathcal{T}_{\mathrm{odd}}
	=
	(\mathcal{V}_{\mathrm{odd}},\ \mathcal{E}_{\mathrm{odd}}),
	\]
	where
	\[
	\mathcal{E}_{\mathrm{odd}}
	=
	\{\, s \to b_{s,x} \mid s \in \mathcal{S},\ b_{s,x} \in B_s \,\}
	\ \cup \
	\{\text{merging edges, automatically introduced according to the connection rules}\}.
	\]
	
	The construction guarantees the following properties:
	
	\begin{enumerate}
		\item Each odd node (except the root $1$) has exactly one parent;
		\item If two branch-generation expressions yield the same odd number, they represent the same node (uniqueness);
		\item There are no cycles among odd nodes;
		\item Each odd node can reach the root $1$ in a finite number of steps along its unique parent chain;
		\item Every odd node is either a starting point or a branch node of some starting point;
		\item The connection rules ensure that the structures generated from all starting points eventually merge into a single rooted tree.
	\end{enumerate}
	
	Hence, $\mathcal{T}_{\mathrm{odd}}$ is a well-defined odd-numbered directed tree rooted at $1$.
	\section{Main Theorem of the Odd-Numbered Tree Model and Rigorous Proof}
	
	This section presents the core main theorem of the Collatz reverse odd-numbered tree model. 
	Using structural arguments, algebraic derivations, and uniqueness of node representations, 
	we prove that the tree model in the odd domain is complete, unique, acyclic, 
	and that each odd node eventually reaches the root node $1$ in a finite number of steps.
	
	\subsection{Main Theorem (Completeness and Uniqueness of the Odd Tree Model)}
	
	\begin{theorem}[Main Theorem of the Odd Tree Model]
		The odd-numbered tree model
		\[
		\mathcal{T}_{\mathrm{odd}}=(\mathcal{V}_{\mathrm{odd}},\ \mathcal{E}_{\mathrm{odd}})
		\]
		satisfies the following properties:
		
		\begin{enumerate}[label=(\arabic*)]
			\item \textbf{Completeness:}  
			All positive odd numbers appear in the node set of $\mathcal{T}_{\mathrm{odd}}$, i.e.,
			\[
			\mathcal{V}_{\mathrm{odd}}=\{1,3,5,\dots\}.
			\]
			
			\item \textbf{Uniqueness:}  
			Each odd node $v \neq 1$ has exactly one parent node, and no two distinct growth/branch representations $(x,n)$, $(x',n')$ produce the same node $v$.
			
			\item \textbf{Acyclicity:}  
			There exist no directed cycles formed by odd nodes, and every odd node can reach the root $1$ in a finite number of reverse steps.
			
			\item \textbf{Finite Regression:}  
			For any odd node $v$, along its unique parent chain
			\[
			v=v_0 \leftarrow v_1 \leftarrow v_2 \leftarrow \cdots
			\]
			the chain terminates at $1$ in finitely many steps.
		\end{enumerate}
		
		Hence, $\mathcal{T}_{\mathrm{odd}}$ is a complete, directed, acyclic odd-numbered tree rooted at $1$.
	\end{theorem}
	
	\subsection{Overview of the Proof Structure}
	
	The proof is divided into four parts, corresponding to the four properties in the main theorem:
	
	\begin{enumerate}
		\item \textbf{Completeness:} Show that every odd number can be generated from the branch node set of some starting point.
		\item \textbf{Uniqueness:} Prove that no conflicts exist across levels, $x$ values, or different starting points.
		\item \textbf{Acyclicity:} Use a potential function to rigorously show that no odd-numbered cycles exist.
		\item \textbf{Finite Regression:} Demonstrate that the reverse path to $1$ cannot be extended infinitely.
	\end{enumerate}
	
	Each part will be rigorously argued in the following subsections.
	\subsection{Proof of Completeness}
	
	Let $v$ be an arbitrary positive odd number. Define
	\[
	m := 3v+1.
	\]
	Since $m$ is even, there exists a unique integer $t \ge 1$ and an odd number $u$ such that
	\[
	m = 2^{t} u, \qquad u \equiv 1 \pmod{2}.
	\]
	
	We consider the parity of $t$ to construct the corresponding starting point and layer, 
	thus showing that $v$ belongs to the branch node set of some starting point.
	
	\medskip
	\noindent\textbf{Case A: $t$ is even.}
	
	Let $t = 2x$ with $x \ge 1$ (since $t \ge 2$ when even). Write
	\[
	u = 6n-5 \qquad (n\ge 1).
	\]
	This is valid because $m = 2^{2x} u$ and $m \equiv 1 \pmod{3}$ imply
	\[
	u \equiv 2^{-2x} \cdot m \equiv m \equiv 1 \pmod{3},
	\]
	and the set of odd numbers congruent to $1 \pmod{3}$ is precisely $\{6n-5\}_{n \ge 1}$. 
	Thus,
	\[
	m = 2^{2x}(6n-5),
	\]
	and consequently
	\[
	v = \frac{m-1}{3} = \frac{2^{2x}(6n-5)-1}{3} = B_1(n,x),
	\]
	i.e., $v$ is a branch node at layer $x$ generated from the starting point $6n-5 \in \mathcal{S}_1$.
	
	\medskip
	\noindent\textbf{Case B: $t$ is odd.}
	
	Let $t = 2x-1$ with $x \ge 1$. Similarly, write
	\[
	u = 6n-1 \qquad (n\ge 1).
	\]
	Since $2^t \equiv -1 \pmod{3}$ for odd $t$, we have
	\[
	-\,u \equiv 1 \pmod{3} \quad \Longrightarrow \quad u \equiv -1 \equiv 2 \pmod{3},
	\]
	and the set of odd numbers congruent to $2 \pmod{3}$ is precisely $\{6n-1\}_{n \ge 1}$. 
	Hence,
	\[
	m = 2^{2x-1} (6n-1),
	\]
	and
	\[
	v = \frac{m-1}{3} = \frac{2^{2x-1}(6n-1)-1}{3} = B_5(n,x),
	\]
	so $v$ is a branch node at layer $x$ from the starting point $6n-1 \in \mathcal{S}_5$.
	
	\medskip
	
	These two cases cover all possible $t$, i.e., for any positive odd $v$, the $2$-adic valuation of $m = 3v+1$ is either even or odd. 
	Therefore, for any odd $v$, there exists a suitable starting point and layer $(n,x)$ such that
	\[
	v = B_1(n,x) \quad \text{or} \quad v = B_5(n,x).
	\]
	By definition, the branch node set $\mathcal{B}(s)$ contains all layers of starting point $s$, 
	so we have
	\[
	v \in \bigcup_{s \in \mathcal{S}} \mathcal{B}(s).
	\]
	
	Hence, the branch node sets generated from all starting points cover all positive odd numbers, establishing completeness.
	\qed
	\subsection{Proof of Uniqueness}
	
	To prove uniqueness, we need to show:
	
	\begin{enumerate}[label=(\arabic*)]
		\item There do not exist distinct $(x,n,r) \neq (x',n',r')$ such that
		\[
		\frac{2^{2x}(6n-r)-1}{3} = \frac{2^{2x'}(6n'-r')-1}{3}.
		\]
		\item No odd node can simultaneously be a branch node of multiple starting points.
		\item No conflict arises between different layers $x$.
	\end{enumerate}
	
	\begin{lemma}[Cross-layer conflict is impossible]
		If
		\[
		2^{2x}(6n-r) = 2^{2x'}(6n'-r'),
		\]
		then $x = x'$.
	\end{lemma}
	
	\begin{proof}
		Both sides are powers of $2$ multiplied by an odd number. 
		The $2$-adic valuation must be equal:
		\[
		v_2(2^{2x}(6n-r)) = 2x, \qquad
		v_2(2^{2x'}(6n'-r')) = 2x'.
		\]
		Equality implies $2x = 2x' \Rightarrow x = x'$.
	\end{proof}
	
	Fixing $x$, we have
	\[
	6n-r = 6n'-r'.
	\]
	Since $r \in \{1,3,5\}$ are distinct, this implies $r = r'$ and then $n = n'$.
	Thus $(x,n,r)$ uniquely determines $v$, establishing uniqueness.
	
	\medskip
	\begin{remark}
		We can also prove completeness and uniqueness via an algebraic approach using arithmetic progressions.
		
		\medskip\noindent\textbf{Arithmetic progression representation of branch generation functions.}
		
		Define
		\[
		C_x := \frac{4^{x-1}-1}{3}, \qquad x\ge1.
		\]
		Then, for $n\ge1$:
		\[
		B_1(n,x) = \frac{(6n-5)2^{2x}-1}{3} = (8n-7) 4^{x-1} + C_x,
		\]
		\[
		B_5(m,x) = \frac{(6m-1)2^{2x-1}-1}{3} = (4m-1) 4^{x-1} + C_x.
		\]
		
		For fixed $x$:
		
		\begin{itemize}
			\item $B_1(n,x)$ increases by $\Delta_1 = 2 \cdot 4^x$ as $n \mapsto n+1$, with first term $B_1(1,x) = 4^{x-1}+C_x$;
			\item $B_5(m,x)$ increases by $\Delta_5 = 4^x$ as $m \mapsto m+1$, with first term $B_5(1,x) = 3 \cdot 4^{x-1}+C_x$.
		\end{itemize}
		
		Define two families of arithmetic progressions:
		\[
		A_x = \{\, a_x + m(2 \cdot 4^x) \mid m \in \mathbb{Z}_{\ge0}\}, \qquad a_x = \frac{4^x-1}{3},
		\]
		\[
		B_x = \{\, b_x + k \cdot 4^x \mid k \in \mathbb{Z}_{\ge0}\}, \qquad b_x = \frac{10 \cdot 4^{x-1}-1}{3}.
		\]
		
		Let
		\[
		\mathcal{A} = \bigcup_{x\ge1} A_x, \qquad \mathcal{B} = \bigcup_{x\ge1} B_x.
		\]
		
		\begin{enumerate}
			\item All elements of $A_x$ and $B_x$ are odd; thus $\mathcal{A}\cup\mathcal{B}$ contains only positive odd numbers.
			\item For any odd number $n$, let $t := v_2(3n+1)$. Then
			\[
			t \text{ even } \iff n \in A_{t/2}, \qquad
			t \text{ odd } \iff n \in B_{(t+1)/2}.
			\]
			\item Therefore, $\mathcal{A}\cup\mathcal{B}$ covers all positive odd numbers, and each odd number belongs to exactly one $A_x$ or $B_x$, ensuring \emph{uniqueness}.
			\item $\mathcal{A}\cup\mathcal{B}$ contains no even numbers.
		\end{enumerate}
		
		\begin{proof}
			(1) All elements are odd: $a_x = (4^x-1)/3 \equiv 1 \pmod 2$, and the common difference $2\cdot4^x$ is even; similarly $b_x = (10\cdot 4^{x-1}-1)/3 \equiv 1 \pmod 2$ with common difference $4^x$ even. Hence all terms are odd.
			
			(2) $2$-adic valuation:  
			\emph{(i) $n \in A_x$}: $n = a_x + m(2 \cdot 4^x)$. Then
			\[
			3n+1 = 3(a_x + m(2\cdot4^x)) + 1 = 4^x (1+6m),
			\]
			where $1+6m$ is odd, so $v_2(3n+1) = 2x$.
			
			\emph{(ii) $n \in B_x$}: $n = b_x + k \cdot 4^x$. Then
			\[
			3n+1 = 3(b_x + k\cdot 4^x)+1 = 2 \cdot 4^{x-1} (5+6k),
			\]
			where $5+6k$ is odd, so $v_2(3n+1) = 2x-1$.
			
			(3) For any odd $n$, let $t = v_2(3n+1)$. If $t$ is even, $n \in A_{t/2}$; if $t$ is odd, $n \in B_{(t+1)/2}$. The sets $A_x$ and $B_x$ are disjoint across $x$, so each odd number belongs to exactly one set.
			
			(4) By (1), no even numbers are included.
		\end{proof}
		
	\end{remark}
	\subsection{Proof of Unique Parent Nodes}
	
	In the odd-number tree model, we aim to prove: for any non-root odd node \(v \in \mathcal{V}_{\mathrm{odd}}\setminus\{1\}\), its parent node (i.e., the starting point that can produce \(v\)) exists uniquely.
	
	\begin{theorem}[Uniqueness of Parent Nodes]
		\label{thm:unique-parent}
		For any \(v \in \mathcal{V}_{\mathrm{odd}}\setminus\{1\}\), there exists a unique starting point \(s \in \mathcal{S}\) and unique layer parameter \(x \ge 1\) and index \(n \ge 1\) such that
		\[
		v = b_{s,x} \in B_s,
		\]
		and this \(s\) is the unique parent of \(v\) in the graph (i.e., there is exactly one incoming edge to \(v\)).
	\end{theorem}
	
	\begin{proof}
		We consider two cases according to the node type.
		
		\medskip\noindent\textbf{Case 1: \(v\) is a starting point (\(v \in \mathcal{S}\)).}
		
		If \(v\in \mathcal{S}\), it naturally has outgoing edges. However, there might exist some starting point \(s\) and layer \(x\) such that \(b_{s,x} = v\), i.e., a branch node coincides with this starting point, creating an incoming edge to \(v\). To show that at most one such incoming edge exists, we need to prove that there cannot exist two distinct parameter pairs \((s,x) \neq (s',x')\) satisfying
		\[
		b_{s,x} = b_{s',x'} = v.
		\]
		
		This follows directly from the theorem in Appendix B on "disjointness of branch generation functions across layers and classes": different classes \(B_1\) and \(B_5\) do not intersect at any fixed layer, and different layers cannot intersect due to the uniqueness lemma for odd numbers multiplied by powers of 2. Therefore, equality can only hold trivially with \(s = s'\) and \(x = x'\). Hence, any branch generation expression that equals \(v\) has unique parameters, giving at most one incoming edge. If no such \(s\) exists, \(v\) has no incoming edge (appearing only as a root or isolated starting point); if it exists, the incoming edge is unique.
		
		\medskip\noindent\textbf{Case 2: \(v\) is a branch node (\(v \in B_s\) for some \(s\)).}
		
		By definition, branch nodes are given by branch generation functions. For \(v\) to be simultaneously generated by two different starting points (or two distinct parameters within the same class), the algebraic equality
		\[
		\frac{(6n-r) 2^{k}-1}{3} = \frac{(6n'-r') 2^{k'} -1}{3}
		\]
		must hold, where \(k \in \{2x, 2x-1\}\), \(k' \in \{2x', 2x'-1\}\), and \(r,r' \in \{1,3,5\}\) indicate the modulo 6 class of the starting point. Multiplying both sides by 3 and adding 1 gives
		\[
		(6n-r) 2^{k} = (6n'-r') 2^{k'}.
		\]
		
		The basic lemma from Appendix B (uniqueness of an odd number multiplied by a power of 2) applies here: if \(u 2^A = v 2^B\) with odd \(u,v\), then \(A=B\) and \(u=v\). Hence \(k=k'\) and \(6n-r = 6n'-r'\), giving \(n=n'\) and \(r=r'\). This contradicts the assumption \((n,k) \neq (n',k')\). Therefore, no two distinct starting points/layer parameters can generate the same branch node \(v\).
		
		Thus, if \(v \in B_s\), its generating starting point \(s\) and parameters \((x,n)\) are unique, so \(s\) is the unique incoming edge source of \(v\) in the graph.
		
		\medskip
		
		Both cases show that for any non-root odd node \(v\), its incoming edge (parent node) either does not exist (for a pure starting point not targeted by any branch) or exists and is unique. Therefore, the uniqueness of parent nodes is established.
	\end{proof}
	\subsection{Rigorous Proof of Acyclicity and Finite Termination}
	
	In this section, we rigorously prove the acyclicity and finite termination of the odd-number tree (i.e., any odd node reaches the root node $1$ in finitely many steps) under the following explicit model assumptions.
	
	\paragraph{Existence and Uniqueness Conditions of Representations}
	
	To define the predecessor nodes that any odd node $y$ must reach in the reverse process, we use the following two types of generation formulas:
	\[
	y_1=(4n-1)4^{x-1}+\frac{4^{x-1}-1}{3},\qquad
	y_2=(8n-7)4^{x-1}+\frac{4^{x-1}-1}{3},
	\]
	where $(x,n)\in\mathbb{Z}_{>0}^2$.  
	All subsequent structural conclusions about the odd-number tree—including acyclicity, potential function decrease, and eventual regression to $1$—depend on the following two basic assumptions regarding these representations.
	
	\medskip
	\noindent
	\textbf{Assumption A (Existence).}
	For each odd $y>1$, there exists at least one pair of positive integers $(x,n)$ satisfying one of the above generation formulas, so that $y$ can be viewed as a branch node generated from some growth node.  
	If some odd numbers do not satisfy this existence condition, the following conclusions hold only for those that do.
	
	\smallskip
	\noindent
	\emph{Remark:} Assumption A follows directly from the \emph{coverage property of branch node sets}. Since the branch node sets are known to cover all positive odd numbers, every odd $y$ appears as a branch node of some growth node and thus satisfies a generation formula.
	
	\medskip
	\noindent
	\textbf{Assumption B (Uniqueness).}
	In each backward step, all legal representations $(x,n)$ may be chosen; our conclusions must hold for all possible choices, i.e., we are proving a strong form that is valid along all legal paths.  
	Equivalently, each odd $y$ must have a unique $(x,n)$, so that its parent node is unique.
	
	\smallskip
	\noindent
	\emph{Remark:} Assumption B is guaranteed rigorously by the \emph{uniqueness of branch nodes}. In the odd-number tree, each odd number appears exactly once in the branch node set, hence the corresponding $(x,n)$ is unique, and thus by
	\[
	F(y) \in \{6n-1, 6n-5\}
	\]
	the parent node is also unique.
	
	\medskip
	Together, Assumptions A and B ensure:
	\[
	\boxed{
		\text{Each odd node $y>1$ has exactly one unique reverse edge; the construction of the odd-number tree is well-defined at every step.}
	}
	\]
	
	This provides a complete and unique structural basis for all subsequent proofs of acyclicity, finite descent, and eventual regression to the root node $1$.
	
	\medskip
	Under these rules, let $p = 4^{x-1}$. To handle all cases uniformly, we define the following potential function.
	
	\begin{definition}[Potential Function]
		For any legal state (i.e., $y$ obtained via some legal $(x,n)$), define
		\[
		\Phi\big(y,(x,n)\big) := \big(A(y,(x,n)),\,y\big),
		\]
		where
		\[
		A(y,(x,n)) :=
		\begin{cases}
			v_2(n), & \text{if } x=1 \text{ and current node is } Y_1\ (y=4n-1),\\[1ex]
			0, & \text{otherwise (including $x \ge 2$ or current node is } Y_2).
		\end{cases}
		\]
		Compare two potential values $(a_1,b_1),(a_2,b_2)$ lexicographically:
		\[
		(a_1,b_1) < (a_2,b_2) \quad\iff\quad
		(a_1<a_2) \text{ or } (a_1=a_2 \text{ and } b_1<b_2).
		\]
	\end{definition}
	
	\begin{lemma}[Strict Single-Step Descent]
		\label{lem:single-step}
		Let the current state be $(y,(x,n))$ (satisfying Assumption A and using any legal representation). Then for any legal successor state $(y',(x',n'))$, we have (except for the terminal state $y=1$)
		\[
		\Phi(y',(x',n')) < \Phi(y,(x,n)).
		\]
	\end{lemma}
	
	\begin{proof}
		We consider all cases according to the current state type:
		
		\paragraph{Case I: $x\ge 2$ (i.e., $p = 4^{x-1} \ge 4$).}
		Here $A(y,(x,n)) = 0$. By algebraic inequalities (depending on $Y_1$ or $Y_2$),
		\[
		y' \le \frac{3y}{2p} + \frac12 \le \frac{3y}{8} + \frac12 < y,
		\]
		so the second component $y$ strictly decreases, and $A'=0$, hence $\Phi(y')<\Phi(y)$ lexicographically.
		
		\paragraph{Case II: $x=1$ and current node is $Y_2$ ($y=8n-7$).}
		If $n=1$, then $y=1$ is terminal; if $n\ge2$, the successor $y'=6n-5<y$ (since $y-y'=2n-2\ge2$), and $A=A'=0$, hence $\Phi(y')<\Phi(y)$.
		
		\paragraph{Case III: $x=1$ and current node is $Y_1$ ($y=4n-1$).}
		Here $A(y)=v_2(n)\ge1$. Consider two subcases:
		\begin{itemize}
			\item If the successor can still be written as $Y_1$ (i.e., $y'=4n'-1$ for some integer $n'$), then
			\[
			4n'-1 = 6n-1 \quad\Longrightarrow\quad n' = \frac{3}{2} n.
			\]
			To be integral, $n$ must be even, and
			\[
			v_2(n') = v_2\left(\frac{3}{2} n\right) = v_2(n)-1.
			\]
			Thus the first component strictly decreases (from $v_2(n)$ to $v_2(n)-1$), so $\Phi(y')<\Phi(y)$ lexicographically.
			
			\item If the successor cannot be written as $Y_1$, then $A(y')=0<A(y)$ (since $A(y)\ge1$), hence $\Phi(y')<\Phi(y)$ lexicographically.
		\end{itemize}
		
		All three cases cover all possibilities, proving strict single-step descent.
	\end{proof}
	
	By Lemma \ref{lem:single-step}, the potential function $\Phi$ strictly decreases at each step and takes values in a well-ordered set of non-negative integers (lexicographically), so no infinite strictly decreasing chain exists.
	
	\begin{proposition}[Finiteness of consecutive $p=1$, $Y_1$ patterns]
		Any consecutive chain of $p=1$ and $Y_1$ type can last only finitely many steps: if at some point $v_2(n)=a$, the pattern lasts at most $a$ steps.
	\end{proposition}
	
	\begin{proof}
		This follows from the analysis in Case III: each step in this pattern decreases $v_2(n)$ by 1; since $v_2$ is a non-negative integer, it cannot decrease indefinitely.
	\end{proof}
	
	\begin{theorem}[No Nontrivial Cycles (Acyclicity)]
		Under Assumptions A and B, the odd-number tree contains no directed cycles except the trivial $1\to1$ loop.
	\end{theorem}
	
	\begin{proof}
		Suppose a nontrivial cycle $y_1 \to y_2 \to \cdots \to y_k \to y_1$ exists. Following the cycle from any starting node, Lemma \ref{lem:single-step} implies the potential $\Phi$ strictly decreases at each step. But returning to the starting node implies the potential must be equal, a contradiction. Hence, no nontrivial cycle exists.
	\end{proof}
	
	\begin{theorem}[Finite Termination]
		Under Assumptions A and B, for any odd $y>1$, there exists a finite integer $T$ such that $F^{T}(y) = 1$.
	\end{theorem}
	
	\begin{proof}
		Suppose the set of counterexamples
		\[
		S := \{y>1 \mid F^t(y) \neq 1 \text{ for all } t\ge 0\}
		\]
		is nonempty. Take its minimal element $y_0$. By the “finite descent” property (from Lemma \ref{lem:single-step} and finiteness of consecutive $p=1,Y_1$ patterns), there exists finite $t$ such that
		\[
		y' := F^t(y_0) < y_0.
		\]
		But $y_0$ is minimal in $S$, so $y' \notin S$, hence there exists $s$ such that $F^s(y')=1$. Therefore $F^{t+s}(y_0)=1$, a contradiction. Thus $S=\varnothing$, proving the theorem.
	\end{proof}
	
	\begin{theorem}[Finite Termination (Main Theorem)]
		Under Assumptions A and B, for any odd $y>1$, there exists finite $T$ such that $F^{T}(y)=1$.
	\end{theorem}
	
	\begin{proof}
		If there were counterexamples
		\[
		S = \{y>1: F^t(y) \neq 1 \ \forall t\},
		\]
		let $y_0 = \min S$. By the finite descent lemma, there exists $t$ such that $y' = F^t(y_0) < y_0$. Hence $y' \notin S$, so its orbit reaches $1$, and thus $y_0$ also reaches $1$, a contradiction. Therefore $S$ is empty.
	\end{proof}
	\subsection{Proof of the Main Theorem}
	
	The above four sections fully cover the four components of the main theorem. Therefore, the main theorem of the odd-number tree model is completely established: 
	the odd-number tree is complete, unique, acyclic, and ensures that all odd nodes eventually reach $1$.
	
	\section{Potential Function and Node Regression}
	
	In this section, we construct a strictly monotonically decreasing integer potential function, 
	which is used to prove that in the tree model (both the odd part and the full model), the backward regression process of all nodes terminates in finitely many steps, 
	ultimately returning to the root node $1$. This potential function is also used to prove acyclicity and serves as the foundation for the rigorous structure of the tree.
	
	\subsection{Construction Rules of the Potential Function and Inclusion of Even Numbers}
	
	The potential function is constructed based on the generation formulas of branch nodes. For any branch node \(y\), if it is produced at the $x$-th position in the branch sequence of some odd starting node 
	\(s \in \{6n-1,6n-5\}\), there exists a unique positive integer pair 
	\((x,n)\) satisfying one of the following:
	\[
	y_1 = (4n-1)4^{x-1} + \frac{4^{x-1}-1}{3}, \qquad
	y_2 = (8n-7)4^{x-1} + \frac{4^{x-1}-1}{3}.
	\]
	Correspondingly, substituting the obtained $n$ into the starting node expression yields a new odd number:
	\[
	y' = 
	\begin{cases}
		6n-1, & \text{if } y=y_1,\\[2mm]
		6n-5, & \text{if } y=y_2.
	\end{cases}
	\]
	This describes the backward regression path from a branch node to its odd starting node. 
	The branch node generation rule implies:
	\[
	\text{Each branch node is uniquely generated by some growth node (i.e., an even node).}
	\]
	Growth nodes cover all positive even numbers and in the forward process eventually descend to some odd node. Therefore:
	
	\begin{itemize}
		\item The backward path of any even node is completely contained within the backward path of its corresponding odd node;
		\item The backward path of an odd node necessarily reaches the root node $1$, so the backward path of an even node is also finite and returns to $1$.
	\end{itemize}
	
	This implies that all positive integers (odd and even) are eventually collected by the same root node $1$ in the backward process.
	
	\subsection{Problem Setup and Notation}
	
	Let $x,n \in \mathbb{Z}_{>0}$ and denote $p = 4^{x-1}$ (so $p=1$ or $p\ge4$). Define two types of representations:
	\[
	\begin{aligned}
		Y_1:&\quad y = (4n-1)p + \frac{p-1}{3}, \qquad \text{with successor } N_1 = 6n-1,\\[1ex]
		Y_2:&\quad y = (8n-7)p + \frac{p-1}{3}, \qquad \text{with successor } N_2 = 6n-5.
	\end{aligned}
	\]
	
	Algorithmic rules:
	
	\begin{enumerate}
		\item Given the current odd number $y$, enumerate $x$ from small to large (i.e., $p=1,4,16,\dots$);
		\item For a fixed $p$, first attempt to write $y$ as $Y_1$: if a positive integer $n$ satisfies the equality, take $N_1=6n-1$ as the next step; otherwise, attempt $Y_2$: if $Y_2$ holds, take $N_2=6n-5$ as the next step; if neither holds for this $p$, increase $p$ and continue;
		\item If $y=1$, stop.
	\end{enumerate}
	
	Note: The above rules are a common implementation strategy; many arguments do not rely on the specific tie-break choice (any legal choice also works). Later we will show that for any (legal) choice, the orbit reaches $1$ in finitely many steps.
	
	For clarity, introduce the following notation and function:
	\[
	F(y) = 
	\begin{cases}
		6n-1, & \text{if $y$ is represented as $Y_1$ by some $(x,n)$},\\[1ex]
		6n-5, & \text{if $y$ is represented as $Y_2$ by some $(x,n)$},
	\end{cases}
	\]
	(In multivalued cases, $F(y)$ can be regarded as a candidate set; if a candidate is fixed for each $y$, $F$ becomes a function.)
	
	Additionally, denote $v_2(m)$ as the $2$-adic valuation of a positive integer $m$ (i.e., $2^{v_2(m)} \parallel m$).
	\subsection{Key Basic Inequalities}
	
	\begin{lemma}[Strict Decrease Due to High Powers]
		If the current representation uses $p \ge 4$ (i.e., $x \ge 2$), then regardless of whether the current state is $Y_1$ or $Y_2$, its successor is strictly less than the current value $y$. More precisely:
		\[
		\begin{aligned}
			&\text{If } y = (4n-1)p + \frac{p-1}{3},\quad N_1 = 6n-1 \le \frac{3y}{2p} + \frac12,\\[2mm]
			&\text{If } y = (8n-7)p + \frac{p-1}{3},\quad N_2 = 6n-5 \le \frac{3y}{4p} + \frac14.
		\end{aligned}
		\]
		Hence, when $p \ge 4$, the right-hand side is strictly less than $y$.
	\end{lemma}
	
	\begin{proof}
		Take $Y_1$ as an example. From $(4n-1)p \le y$ we get $n \le \frac{y/p + 1}{4}$. Substituting into $N_1 = 6n-1$ gives the stated upper bound. The $Y_2$ case is similar. For $p \ge 4$, direct comparison shows the right-hand side $< y$.
	\end{proof}
	
	\begin{lemma}[Two Subcases for $p=1$]
		When $p=1$ ($x=1$):
		\begin{itemize}
			\item If $y = 4n-1$ ($Y_1$), then $F(y) = 6n-1 > y$ (strict increase).
			\item If $y = 8n-7$ ($Y_2$), then $F(y) = 6n-5$. When $n=1$, $F(y)=1$ (fixed point); when $n \ge 2$, $F(y) = 6n-5 < 8n-7 = y$ (strict decrease).
		\end{itemize}
	\end{lemma}
	
	\begin{proof}
		Direct substitution and calculation of differences.
	\end{proof}
	
	\subsection{No Infinite “Lift”: Exhaustion of $v_2$}
	
	\begin{lemma}[Exhaustion of $v_2$ in the $p=1, Y_1$ Pattern]
		Suppose at some step we have $y_k = 4n_k - 1$ (i.e., $p=1, Y_1$), and the successor $y_{k+1} = 6n_k-1$ can still be written as $4n_{k+1}-1$. Then
		\[
		n_{k+1} = \frac{3}{2} n_k, \qquad v_2(n_{k+1}) = v_2(n_k) - 1.
		\]
		Hence, if at some initial index $k_0$ we have $v_2(n_{k_0}) = a$, this pattern can persist for at most $a$ consecutive steps, and cannot continue indefinitely.
	\end{lemma}
	
	\begin{proof}
		From $4n_{k+1}-1 = 6n_k-1$ we get $n_{k+1} = \tfrac{3}{2} n_k$. To maintain integrality, $2 \mid n_k$, write $n_k = 2^{a_k} m_k$ ($m_k$ odd). Then $n_{k+1} = 3 \cdot 2^{a_k-1} m_k$, so $v_2(n_{k+1}) = a_k - 1$. Repeating this argument gives the conclusion.
	\end{proof}
	
	\begin{remark}
		Intuitively, each “increase” in the $p=1, Y_1$ pattern consumes one factor of $2$. Since the number of such factors is finite, the increase cannot continue indefinitely.
	\end{remark}
	
	\subsection{Potential Function (Ordered Pair) and Complete Exhaustion for Single-Step Strict Decrease}
	
	To consolidate the above fragmented facts into a single invariant valid at every step, we construct the potential function.
	
	\begin{definition}[Potential Function $\Phi$]
		For a state $(y,(x,n))$ (where $y$ is currently represented by $(x,n)$), define
		\[
		\Phi(y,(x,n)) := \big(A(y,(x,n)),\, y\big),
		\]
		where
		\[
		A(y,(x,n)) =
		\begin{cases}
			v_2(n), & \text{if } x=1 \text{ and } y=4n-1 \ (\text{i.e., } p=1,Y_1),\\[1ex]
			0, & \text{otherwise (including } x \ge 2 \text{ or } y = 8n-7).
		\end{cases}
		\]
		Compare potential by lexicographic order: $(a_1,b_1) < (a_2,b_2)$ if and only if $a_1 < a_2$ or $a_1 = a_2$ and $b_1 < b_2$.
	\end{definition}
	
	Clearly, the range of $\Phi$ is a set of non-negative integer pairs, which is well-ordered in lexicographic order (no infinite strictly decreasing chains).
	
	\begin{proposition}[Single-Step Strict Decrease — Complete Exhaustion of Cases]
		For any valid current state $(y,(x,n))$ (i.e., $(x,n)$ represents $y$ as $Y_1$ or $Y_2$), take any valid next state $(y',(x',n'))$ (i.e., some legal representation and corresponding $N$). Then, except for the terminal state $y=1$,
		\[
		\Phi(y',(x',n')) < \Phi(y,(x,n)).
		\]
		In other words, the potential strictly decreases at every step, and this conclusion holds without exception for all parameter configurations.
	\end{proposition}
	
	\begin{proof}
		We exhaust all possible types of the current state (three main cases):
		
		\medskip
		
		\noindent\textbf{Case A: Current $p \ge 4$ ($x \ge 2$).}  
		Then $A(y,(x,n))=0$. By the high-power lemma, the next $y' < y$. Hence $A'=0$ and $y'<y$, so lexicographically $\Phi(y') < \Phi(y)$.
		
		\medskip
		
		\noindent\textbf{Case B: Current $p=1$ and $Y_2$ ($y=8n-7$).}  
		If $n=1$, then $y=1$ (terminal). If $n \ge 2$, then $y' = 6n-5 < y$, and $A = A' = 0$, hence $\Phi(y') < \Phi(y)$.
		
		\medskip
		
		\noindent\textbf{Case C: Current $p=1$ and $Y_1$ ($y=4n-1$).}  
		Then $A(y) = v_2(n) \ge 1$. Two subcases:
		\begin{itemize}
			\item If the successor can still be written as $Y_1$ (there exists $n'$ such that $y' = 4n'-1$), then $n' = \tfrac{3}{2} n$, so $v_2(n') = v_2(n) - 1$. Hence the first component strictly decreases ($A' = A-1$), so lexicographically $\Phi(y') < \Phi(y)$ regardless of $y'$.
			\item If the successor cannot be written as $Y_1$ (pattern breaks: the successor appears as $p' \ge 4$ or $Y_2$), then $A' = 0 < A$, so the first component strictly decreases, and thus $\Phi(y') < \Phi(y)$.
		\end{itemize}
		
		These three cases cover all possibilities. The proof is complete.
	\end{proof}
	\subsection{Global Termination and Absence of Nontrivial Cycles}
	
	\begin{theorem}[Global Termination]
		For any valid initial $y_0>1$, along any trajectory following valid choices (at each step, choose any valid $(x,n)$ and take the corresponding $N$), the trajectory reaches $1$ in a finite number of steps.
	\end{theorem}
	
	\begin{proof}
		By the previous proposition, the potential $\Phi$ strictly decreases at each step, and the range of $\Phi$ is well-ordered (non-negative integer pairs under lexicographic order). Therefore, there cannot exist an infinite strictly decreasing chain. Hence, any trajectory cannot extend indefinitely without termination and must reach a terminal state in a finite number of steps. The unique terminal state that cannot decrease further and satisfies the representation rules is $y=1$ (corresponding to $Y_2$ with $n=1$). Therefore, any trajectory reaches $1$ in finitely many steps.
	\end{proof}
	
	\begin{corollary}[Absence of Nontrivial Cycles]
		There exists no directed cycle (periodic point) except the trivial self-loop $1 \to 1$.
	\end{corollary}
	
	\begin{proof}
		If a cycle of length $\ge 1$ not entirely consisting of $1$ exists, then along the cycle the potential would return to its original value after one round, contradicting the strict single-step decrease. Therefore, no such cycle exists.
	\end{proof}
	
	\subsection{Summary Table of Exhaustive Cases (for Reference and Implementation)}
	
	The following table summarizes the exhaustive possibilities for the current state and the next step, along with the reason for the potential decrease (concise version):
	
	\begin{longtable}{@{}p{4.2cm} p{7.0cm} p{4.0cm}@{}}
		\toprule
		Current Type & Possible Next Form and Integrality Conditions & Reason for $\Phi$ Decrease \\
		\midrule
		\endhead
		
		$p\ge4$ (any $Y_1$ or $Y_2$) & Next step may be $p'=1$ ($Y_1$ or $Y_2$) or $p'\ge4$; in any case, $y'<y$ by the high-power inequality & Second component $y$ decreases, $A=0 \Rightarrow \Phi$ decreases \\[2mm]
		
		$p=1,\; Y_2,\; n=1$ & $y=1$: terminal & Terminal state \\[2mm]
		
		$p=1,\; Y_2,\; n\ge2$ & Any valid next representation; $y' = 6n-5 < y$ & Second component $y$ decreases, $A=0 \Rightarrow \Phi$ decreases \\[2mm]
		
		$p=1,\; Y_1$ and successor still $Y_1$ & $n$ must be even, $n' = \frac32 n$; $v_2(n)$ decreases by 1 per step & First component $A = v_2(n)$ decreases ($A \mapsto A-1$) $\Rightarrow \Phi$ decreases \\[2mm]
		
		$p=1,\; Y_1$ and successor not $Y_1$ & Successor is $p'\ge4$ or $Y_2$ (if $Y_2$ and $n'\ge2$, then $y'$ decreases; if $n'=1$, then reaches 1) & $A$ drops to 0 or $y$ decreases $\Rightarrow \Phi$ decreases \\
		
		\bottomrule
	\end{longtable}
	
	\begin{lemma}[Bounded Strong Shrink per Step]
		\label{lem:strong-shrink}
		Let the current $y \ge 2$, and suppose $p \ge 4$ (i.e., $x \ge 2$) is used in this step. Then the successor $y'$ satisfies
		\[
		y' \le \left\lfloor \frac{3y}{8} \right\rfloor + 1.
		\]
		In particular, $y' < y$.
	\end{lemma}
	
	\begin{proof}
		If the current state is represented in $Y_1$ form:
		\[
		y = (4n-1)p + \frac{p-1}{3},
		\]
		then $(4n-1)p \le y$, which gives
		\[
		n \le \frac{y/p + 1}{4}.
		\]
		Hence the successor
		\[
		y' = 6n-1 \le \frac{3y}{2p} + \frac12 \le \frac{3y}{8} + \frac12.
		\]
		Taking the floor gives $y' \le \big\lfloor \frac{3y}{8} \big\rfloor + 1$. The $Y_2$ case is similar and yields an even smaller upper bound. Thus the conclusion holds.
	\end{proof}
	\subsection{Conservative Global Step Bound}
	
	This section provides a conservative upper bound on the number of steps for all possible trajectories. Intuitively, we divide the iteration into several \emph{phases}, each bounded by a ``strong shrink''—i.e., a step using $p\ge4$ that produces significant compression. We provide (1) an upper bound on the number of steps within each phase, and (2) the number of strong shrinks required to reduce the value to $1$. Multiplying these gives a conservative global bound.
	
	\begin{lemma}[Upper Bound on Consecutive $p=1,Y_1$ Steps within a Phase]
		\label{lem:phase-length}
		Let $y^{(0)}$ be the current value at the start of a phase, and define
		\[
		L := \big\lfloor \log_2 y^{(0)} \big\rfloor + 1.
		\]
		Then, within this phase (i.e., before the next strong shrink occurs), there can be at most $L$ consecutive ``rising'' steps of type $p=1$ and $Y_1$. In other words, the number of steps from the start of a phase until the next strong shrink is bounded above by $L$.
	\end{lemma}
	
	\begin{proof}
		Within a phase, any step that is $p=1,Y_1$ with a successor still of type $p=1,Y_1$ corresponds to the transformation $n \mapsto n' = \tfrac{3}{2} n$, and each step reduces $v_2(n)$ by $1$. Thus, the maximal number of consecutive steps of this type is bounded by the $2$-adic order $v_2(n)$ at the start of the phase. To obtain a conservative bound in terms of $y^{(0)}$, observe that if $y=4n-1$, then
		\[
		n \le \frac{y+1}{4} \le y,
		\]
		hence
		\[
		v_2(n) \le \lfloor \log_2 n \rfloor \le \lfloor \log_2 y \rfloor \le \lfloor \log_2 y^{(0)} \rfloor.
		\]
		Including a possible extra step, we take the conservative upper bound
		\[
		L = \lfloor \log_2 y^{(0)} \rfloor + 1.
		\]
		Thus, the total number of steps within a phase does not exceed $L$.
	\end{proof}
	
	\begin{lemma}[Upper Bound on the Number of Strong Shrinks]
		\label{lem:number-strong}
		Let $y_0 \ge 2$ be the initial value. If after $m$ strong shrinks (i.e., $m$ steps with $p \ge 4$) the value has not yet reached $1$, then
		\[
		y^{(m)} \le \left(\frac{3}{8}\right)^m y_0 + \frac{8}{5},
		\]
		where $y^{(m)}$ denotes the value immediately after the $m$-th strong shrink. Therefore, if
		\[
		m \ge \Big\lceil \log_{8/3} y_0 \Big\rceil + 1,
		\]
		then $y^{(m)} \le 1$, i.e., at most
		\[
		K := \Big\lceil \log_{8/3} y_0 \Big\rceil + 1
		\]
		strong shrinks are needed to reduce the value to $\le 1$.
	\end{lemma}
	
	\begin{proof}
		By Lemma~\ref{lem:strong-shrink}, a strong shrink reduces the value from $y$ to
		\[
		y' \le \frac{3y}{8} + \frac12.
		\]
		Applying this iteratively $m$ times and treating the additive constant as a geometric series gives
		\[
		y^{(m)} \le \left(\frac{3}{8}\right)^m y_0 + \frac12 \sum_{j=0}^{m-1} \left(\frac{3}{8}\right)^j
		= \left(\frac{3}{8}\right)^m y_0 + \frac12 \cdot \frac{1-(3/8)^m}{1-3/8}.
		\]
		Bounding the second term,
		\[
		\frac12 \cdot \frac{1}{1-3/8} = \frac12 \cdot \frac{1}{5/8} = \frac{8}{10} = \frac{4}{5}.
		\]
		We therefore take the conservative estimate
		\[
		y^{(m)} \le \left(\frac{3}{8}\right)^m y_0 + \frac{4}{5} < \left(\frac{3}{8}\right)^m y_0 + \frac{8}{5}.
		\]
		To ensure the right-hand side $\le 1$, it suffices to require $\left(\frac{3}{8}\right)^m y_0 < 1$, which holds for $m > \log_{8/3} y_0$. Taking the ceiling and adding $1$ to absorb the constant term gives the stated $K$.
	\end{proof}
	
	\begin{theorem}[Conservative Global Step Bound]
		\label{thm:global-bound}
		For any initial value $y_0 \ge 2$, the number of steps $T(y_0)$ to reach $1$ satisfies the conservative bound
		\[
		T(y_0) \le L(y_0) \cdot K(y_0),
		\]
		where
		\[
		L(y_0) := \big\lfloor \log_2 y_0 \big\rfloor + 1, 
		\qquad
		K(y_0) := \Big\lceil \log_{8/3} y_0 \Big\rceil + 1.
		\]
		Hence $T(y_0) = O\big((\log y_0)^2\big)$, and this bound holds for all possible trajectories.
	\end{theorem}
	
	\begin{proof}
		Divide the iteration into phases, defined as intervals between consecutive strong shrinks (if the initial step is a strong shrink, the first phase starts after it). Within each phase, by Lemma~\ref{lem:phase-length}, there are at most $L(y_0)$ steps (conservatively, since $y$ within the phase does not exceed $y_0$, giving $v_2(n) \le \lfloor \log_2 y_0 \rfloor$). Thus, the total number of steps per phase is $\le L(y_0)$.
		
		By Lemma~\ref{lem:number-strong}, at most $K(y_0)$ strong shrinks are required to reduce the value to $\le 1$. Multiplying the upper bounds on phase length and number of phases gives
		\[
		T(y_0) \le L(y_0) \cdot K(y_0).
		\]
		Since both $L$ and $K$ are $O(\log y_0)$, we obtain $T(y_0) = O((\log y_0)^2)$, completing the proof.
	\end{proof}
	
	\begin{remark}
		This bound is conservative: in actual trajectories, the value often decreases faster (e.g., strong shrinks may occur earlier, or $v_2(n)$ may be small, shortening the phase). The main value of this bound is that it provides a uniform guarantee for \emph{all} possible trajectories, including the worst-case scenario. For tighter bounds for a specific starting value, one may track the evolution of $v_2(n)$ along the trajectory instead of using the conservative estimate.
	\end{remark}
	\subsection{Rigorous Supplement of the Minimal Counterexample Method}
	
	Based on the previous results on existence and uniqueness of representations (Assumptions A and B) and the single-step strict descent lemma, this section rewrites the “minimal counterexample method” into a fully self-contained, directly citable rigorous proof.
	
	For convenience, denote by $F(\cdot)$ the multi-valued map that assigns to each current value all possible successors obtained by choosing any legal representation $(x,n)$. When needed, $F$ can be treated as single-valued by fixing a “selection strategy.” All assertions below hold in the strong sense: for \emph{any} reasonable selection strategy.
	
	\begin{enumerate}
		\item[(A)] (Existence) For every odd $y>1$, there exists at least one $(x,n)\in \mathbb Z_{>0}^2$ such that
		\[
		y=(4n-1)4^{x-1}+\frac{4^{x-1}-1}{3} \quad \text{or} \quad
		y=(8n-7)4^{x-1}+\frac{4^{x-1}-1}{3}.
		\]
		\item[(B)] (Uniqueness / Well-definedness) Within the branch node set, the representation of each odd $y$ (if it exists) is unique, so the corresponding parent node in the model is well-defined.
		\item Any step may choose an arbitrary legal representation; all conclusions below hold for any such choice (i.e., in the strong form).
	\end{enumerate}
	
	\begin{lemma}[Single-Step Strict Descent (Recap)]
		At any legal current state $(y,(x,n))$, any legal successor $(y',(x',n'))$ (except $y=1$) strictly decreases a clearly defined potential function $\Phi$ (e.g., lexicographic $\Phi=(A,y)$ or $\Phi = \log_4 y + Cx$):
		\[
		\Phi(y',(x',n')) < \Phi(y,(x,n)).
		\]
	\end{lemma}
	
	\begin{proof}
		Directly cite the case-by-case proof given in the main text: strong shrink for $p\ge4$, decrease for $p=1,Y_2$, and exhaustion of $v_2(n)$ leading to decrease for $p=1,Y_1$.  
		See the main text section ``Single-Step Strict Descent.''
	\end{proof}
	
	\begin{lemma}[Finite Descent]
		\label{lem:finite-descent}
		For any odd $y>1$, under any legal selection strategy, there exists a finite positive integer $t$ and a legal path starting from $y$ (choosing arbitrary legal representations at each step) such that after $t$ steps the resulting value $y^{(t)}$ satisfies
		\[
		y^{(t)} < y.
		\]
		In other words, starting from any node, one can reach a strictly smaller node in finitely many steps.
	\end{lemma}
	
	\begin{proof}
		Exhaust all possible representation types of the current $y$ (completeness guaranteed by Assumption A):
		
		\medskip\noindent\textbf{Case 1: Current $p=4^{x-1}\ge4$.}  
		By the strong shrink inequality in the main text, one step yields $y'<y$, so take $t=1$.
		
		\medskip\noindent\textbf{Case 2: Current $p=1$ and $Y_2$ ($y=8n-7$).}  
		If $n=1$ then $y=1$ (termination); if $n\ge2$ then $y'=6n-5<y$, so again $t=1$.
		
		\medskip\noindent\textbf{Case 3: Current $p=1$ and $Y_1$ ($y=4n-1$).}  
		One step may produce $y'=6n-1>y$ (short-term increase). However, by the exhaustion argument for $p=1,Y_1$, each step reduces $v_2(n)$ by $1$, so this pattern can last at most $v_2(n)$ steps. Along any legal path:
		
		\begin{itemize}
			\item Either within finitely many steps a successor falls into $Y_2$ or requires $p'\ge4$, which by Case 1 or 2 yields a strict decrease;
			\item Or the $p=1,Y_1$ pattern is exhausted ($v_2=0$), so the next step cannot remain $Y_1$, necessarily producing a decrease.
		\end{itemize}
		
		Hence, in all possible scenarios, a strict descent occurs in finitely many steps. Thus, there exists finite $t$ such that $y^{(t)}<y$.
	\end{proof}
	
	\begin{theorem}[Minimal Counterexample Method: Finite Regression]
		\label{thm:termination}
		Under Assumptions A, B, and the single-step strict descent lemma, any odd $y>1$ reaches $1$ in finitely many steps. That is, for any $y>1$ there exists a positive integer $T$ such that $F^T(y)=1$.
	\end{theorem}
	
	\begin{proof}
		By contradiction using the minimal counterexample method, let
		\[
		S := \{ y>1 \mid \forall t\ge0, \ F^t(y)\ne 1 \}
		\]
		be the set of odd numbers that cannot reach $1$. If $S=\varnothing$, the theorem holds. Otherwise, let $y_0 = \min S$ (minimal element in the natural order).
		
		By Lemma~\ref{lem:finite-descent}, there exists finite $t\ge1$ and a legal path from $y_0$ yielding $y':=F^t(y_0)$ with $y'<y_0$. By minimality, $y'\notin S$, so there exists $s\ge0$ with $F^s(y')=1$. Thus
		\[
		F^{t+s}(y_0)=1,
		\]
		contradicting $y_0\in S$. Therefore, $S$ is empty and the theorem holds.
	\end{proof}
	
	\begin{remark}
		\begin{enumerate}
			\item The proof relies critically on the lemma ``finite descent from any point,'' so the validity of the minimal counterexample method reduces entirely to the rigorous proof of Lemma~\ref{lem:finite-descent}.
			\item The proof holds for \emph{any} legal selection strategy—Lemma~\ref{lem:finite-descent} considers all possible legal paths (in Case 3 we exhaustively treat all successor scenarios). To restrict to a specific strategy (e.g., always choosing minimal $x$), simply replace “any” with “fixed” at the corresponding point; the proof remains valid.
			\item If using a numeric potential function (e.g., $\Phi(y)=\log_4 y + Cx$), one may replace ``strict descent'' with the discrete decrease of $\Phi$, which quantitatively establishes the existence of a descending step; the minimal counterexample method applies identically.
		\end{enumerate}
	\end{remark}
	\subsection{Conclusion}
	
	In this section, we rigorously established that under the given two types of representations and iteration rules:
	
	\begin{enumerate}
		\item Any orbit starting from an odd integer (or from an even integer divided repeatedly by 2 to reach an odd integer) reaches $1$ in finitely many steps;
		\item There exist no cycles other than the trivial fixed point $1$;
		\item Any consecutive $p=1,Y_1$ “lifting” pattern is exhausted in finitely many steps (by the $v_2$ argument).
	\end{enumerate}
	
	These three facts together, via the construction of a well-ordered potential function, provide evidence for global single-step strict descent and termination. The paper also provides a complete exhaustive table of single-step cases, which is useful for implementation or citation.
	
	\section{Construction and Properties of the Complete Tree Model (Including Even Numbers)}
	
	Building on the odd-number tree model constructed in the previous section, we now introduce even nodes (i.e., growth nodes) to form a directed tree structure containing \emph{all positive integers}. We will show that the inclusion of even nodes does not disrupt any core structural properties of the odd-number tree model and that the odd-number coverage naturally extends to full positive-integer coverage.
	
	\subsection{Definition of the Complete Tree Model}
	
	We retain all odd nodes from the odd-number tree model and, for each odd node $o$, introduce its \emph{growth chain}:
	\[
	o \;\longrightarrow\; 2o \;\longrightarrow\; 4o \;\longrightarrow\; 8o \;\longrightarrow \cdots.
	\]
	
	The complete tree model is defined as follows.
	
	\begin{definition}[Complete Tree Model]
		Given a set of starting nodes $\mathcal{S}$, growth nodes $\mathcal{G}$, and branch nodes $\mathcal{B}$, construct a directed graph containing all positive integers by adding the following types of edges:
		
		\begin{enumerate}
			\item \textbf{Even-growth edges:}  
			For each odd node $o$, add edges
			\[
			o \to 2o, \qquad 2o \to 4o, \qquad 4o \to 8o, \ \ldots
			\]
			ensuring that every even integer can be expressed as a $2^k$ multiple of some odd node.
			
			\item \textbf{Odd-branch edges:}  
			For each node $y$ satisfying the backward condition $y \equiv 1 \pmod{3}$, add the edge
			\[
			\frac{y-1}{3} \to y.
			\]
			
			\item \textbf{Connection rule:}  
			Whenever a branch node coincides with a starting node, merge their structures.
		\end{enumerate}
		
		The resulting directed structure is called the \emph{complete Collatz tree model}, denoted by $\mathcal{T}_{\mathrm{full}}$.
	\end{definition}
	
	\subsection{The Complete Tree Contains the Odd-Number Tree as Its Skeleton}
	
	The odd-number tree model $\mathcal{T}_{\mathrm{odd}}$ consists of starting nodes and branch nodes, with node set
	\[
	V_{\mathrm{odd}} = \{\text{all positive odd integers}\}.
	\]
	
	The additional nodes in the complete tree model come entirely from the even growth chains:
	\[
	V_{\mathrm{even}} = \{2^k o \mid o \text{ is odd},\, k \ge 1\}.
	\]
	
	Hence, the complete tree model satisfies
	\[
	V_{\mathrm{full}} = V_{\mathrm{odd}} \cup V_{\mathrm{even}} = \mathbb{Z}_{>0}.
	\]
	
	The odd-number tree model forms an induced subgraph within the complete tree, and all of its structure remains unchanged by the addition of even nodes.
	\subsection{Directionality and Preservation of Unique Parent Nodes}
	
	After introducing even nodes, each node still has a unique parent:
	
	\begin{enumerate}
		\item For an even node $y = 2^k o$ ($o$ odd):
		\[
		\text{the unique parent is } 2^{k-1}o.
		\]
		
		\item For an odd node satisfying $y \equiv 1 \pmod{3}$:
		\[
		\text{the unique parent is } \frac{y-1}{3}.
		\]
		
		\item For an odd node $y \not\equiv 1 \pmod{3}$ with $y>1$:
		\[
		\text{there is no odd parent, and the even parent } 2^{-1}y \text{ exists only if $y$ is even.}
		\]
		Since $y$ is odd, this case does not occur.
		
		\item For $y=1$, no parent exists.
	\end{enumerate}
	
	Therefore:
	
	\begin{proposition}[Preservation of Unique Parent Nodes]
		Every node in the complete tree model $\mathcal{T}_{\mathrm{full}}$ (except the root node $1$) has a unique parent.
	\end{proposition}
	
	\subsection{Introduction of Even Nodes Does Not Create New Cycles}
	
	Any backward edge involving an even node has the form
	\[
	2^k o \;\longleftarrow\; 2^{k-1}o,
	\]
	and the potential function (as defined in the previous section)
	\[
	\Phi(2^k o) = 2^k o
	\]
	strictly decreases, so no cycles can appear in an even-number chain.
	
	For the odd nodes, the backward process also strictly decreases the potential function:
	\[
	\Phi(y) = 3y, \qquad \Phi\!\left(\frac{y-1}{3}\right) = y-1 < 3y.
	\]
	
	Hence, along any backward path in the complete tree, the potential function always strictly decreases, preventing any cycles.
	
	\begin{theorem}[Acyclicity of the Complete Tree]
		The complete tree model $\mathcal{T}_{\mathrm{full}}$ is acyclic.
	\end{theorem}
	
	\subsection{Completeness: Coverage of All Positive Integers}
	
	\begin{proposition}[Coverage of Positive Integers]
		The complete tree model $\mathcal{T}_{\mathrm{full}}$ covers all positive integers:
		\[
		V_{\mathrm{full}} = \mathbb{Z}_{>0}.
		\]
	\end{proposition}
	
	\begin{proof}
		Take any $y \in \mathbb{Z}_{>0}$.
		
		If $y$ is odd, then $y \in V_{\mathrm{odd}}$ and is covered by the odd-number tree.
		
		If $y$ is even, it can be uniquely written as
		\[
		y = 2^k o, \qquad o \text{ odd},
		\]
		where $o \in V_{\mathrm{odd}}$ and $2^k o \in V_{\mathrm{even}}$.
		Hence, $y$ is also covered by the complete tree.
	\end{proof}
	
	\subsection{The Complete Tree Preserves All Structural Properties of the Odd-Number Tree}
	
	Since each even-number chain is a strictly increasing chain in the forward direction and strictly decreasing in the backward direction, and each even node is uniquely associated with an odd ancestor $o$, the inclusion of even nodes does not alter:
	
	\begin{itemize}
		\item odd-number coverage $\to$ \textbf{positive-integer coverage};
		\item node uniqueness;
		\item unique parent property;
		\item directionality;
		\item acyclicity;
		\item backward paths terminate uniquely at the root $1$;
		\item merging of structures for odd starting nodes still follows the branch condition uniquely.
	\end{itemize}
	
	Therefore:
	
	\begin{theorem}[Preservation of Complete Tree Structure]
		The complete tree model $\mathcal{T}_{\mathrm{full}}$ extends the odd-number tree model $\mathcal{T}_{\mathrm{odd}}$ by adding only the even-number chains, thereby extending coverage from all positive odd integers to all positive integers without altering any other structural properties.
	\end{theorem}
	
	\section{Bijection Between the Collatz Map and the Tree Model}
	\label{sec:collatz-bijection}
	
	In this section, we rigorously prove that \emph{each Collatz orbit corresponds one-to-one to a backward path in the complete tree model constructed above}. This bijection has two directions:
	
	\begin{itemize}
		\item[(1)] (\textbf{Forward}) The forward Collatz iteration sequence of any positive integer $N$,
		\[
		N \mapsto T(N) \mapsto T^2(N) \mapsto \cdots,
		\]
		corresponds to a finite backward path in the tree starting from node $N$ to the root $1$.
		
		\item[(2)] (\textbf{Backward}) Any backward path from a node in the tree
		\[
		n = n_0 \leftarrow n_1 \leftarrow \cdots \leftarrow n_k = 1
		\]
		corresponds to the forward Collatz iteration of some positive integer.
	\end{itemize}
	
	Thus, the tree model provides a \emph{complete and closed representation} of the Collatz iteration.
	
	\subsection{Forward Correspondence: Collatz Iteration $\Rightarrow$ Tree Backward Path}
	
	Recall the Collatz map
	\[
	T(n) =
	\begin{cases}
		n/2, & n \equiv 0 \pmod{2},\\[0.3em]
		3n+1, & n \equiv 1 \pmod{2}.
	\end{cases}
	\]
	
	\begin{lemma}[Correspondence of Even Steps]
		\label{lem:even-step}
		Let $n$ be even. Then the Collatz next step is $n/2$.
		In the tree model, every even node belongs to a growth chain of some odd node $u$:
		\[
		n = 2^k u, \qquad u \text{ is its unique odd ancestor}.
		\]
		Hence $n/2$ corresponds to one backward step along the growth chain.
	\end{lemma}
	
	\begin{proof}
		Every even node in the tree is generated by the growth rule:
		\[
		u \mapsto 2u \mapsto 4u \mapsto \cdots \mapsto 2^k u = n.
		\]
		Reversing one step yields $n/2 = 2^{k-1} u$, which is unique.
		This exactly matches the Collatz even step.
	\end{proof}
	
	\begin{lemma}[Correspondence of Odd Steps]
		If $n$ is odd and $n \ne 1$, the Collatz next step is $3n+1$, which is even.
		In the tree model, the backward odd parent is uniquely given by
		\[
		\frac{3n+1}{2^{v_2(3n+1)}},
		\]
		where $v_2(\cdot)$ denotes the $2$-adic valuation.
	\end{lemma}
	
	\begin{proof}
		The unique parent of an odd node in the tree is determined by the branch condition
		\[
		\frac{m-1}{3} \in \mathbb{N}.
		\]
		If a Collatz odd number $n$ maps to an even number $m=3n+1$, then $n = (m-1)/3$.
		Dividing out all powers of $2$ from $m$ along the growth chain still yields the unique odd ancestor.
		Hence, backward odd steps in the tree correspond exactly to Collatz odd steps.
	\end{proof}
	
	\begin{theorem}[Completeness of Forward Correspondence]
		\label{thm:forward-bijection}
		For any positive integer $N$, its Collatz forward iteration sequence corresponds to a finite backward path in the tree starting from $N$, through a sequence of backward growth and backward branching steps, ending at the root $1$.
	\end{theorem}
	
	\begin{proof}
		Even steps follow Lemma~\ref{lem:even-step}; odd steps follow the previous lemma.
		Since the tree has a unique parent structure (as proven above), the backward path is unambiguous.
		
		Moreover, the potential function strictly decreases along backward steps:
		backward growth reduces the $2$-adic exponent, and backward branching reduces the odd height.
		Hence the backward path is finite.
		
		Therefore, each Collatz orbit is embedded in a backward path in the tree.
	\end{proof}
	\subsection{Backward Correspondence: Tree Backward Path $\Rightarrow$ Collatz Iteration}
	
	\begin{lemma}[Backward Growth Corresponds to Collatz Even Step]
		If in the tree $n = 2m$, and a backward step is $n \to m$,
		then the forward Collatz map satisfies $m \mapsto n$, i.e.,
		$T(m) = n$.
	\end{lemma}
	
	\begin{lemma}[Backward Branch Corresponds to Collatz Odd Step]
		If in the tree a backward step is
		\[
		m \to \frac{m-1}{3},
		\]
		then $\frac{m-1}{3}$ is odd, and the Collatz map satisfies
		\[
		T\Big(\frac{m-1}{3}\Big) = m.
		\]
	\end{lemma}
	
	Combining these two lemmas with the tree structure, we obtain:
	
	\begin{theorem}[Completeness of Backward Correspondence]
		\label{thm:backward-bijection}
		For any node $n$ in the tree, its backward path
		\[
		n = n_0 \leftarrow n_1 \leftarrow \cdots \leftarrow n_k = 1
		\]
		corresponds exactly to the Collatz iteration sequence of some positive integer $n$:
		\[
		n = T^0(n) \mapsto T^1(n) \mapsto \cdots \mapsto T^k(n) = 1.
		\]
	\end{theorem}
	
	\begin{proof}
		Each backward step in the tree consists entirely of backward growth and backward branching.
		Backward growth corresponds to the inverse of a Collatz even step, and backward branching corresponds to the inverse of a Collatz odd step.
		
		Reversing the order of these backward steps yields the forward Collatz iteration steps.
	\end{proof}
	
	\subsection{Main Bijection Theorem}
	
	\begin{theorem}[Collatz--Tree Model Bijection Theorem]
		\label{thm:collatz-bijection}
		The complete tree model constructed from odd starting points, growth rules, and branch rules
		establishes a strict two-way bijection with the Collatz map:
		
		\begin{enumerate}
			\item[(1)] Each Collatz iteration sequence corresponds to exactly one backward path in the tree.
			\item[(2)] Each backward path in the tree corresponds to exactly one Collatz iteration sequence.
			\item[(3)] A Collatz orbit terminates at $1$ if and only if the tree backward path terminates at the root node $1$.
			\item[(4)] Collatz has no nontrivial cycles $\Longleftrightarrow$ the tree has no directed cycles except at the root (as previously proven).
		\end{enumerate}
		
		Hence, the tree model provides a complete, closed, and conflict-free representation of the Collatz map.
	\end{theorem}
	
	\begin{proof}
		Follows directly from Theorem~\ref{thm:forward-bijection} and
		Theorem~\ref{thm:backward-bijection}.
	\end{proof}
	\section{Main Theorem: Completeness—Closure—Acyclicity}
	\label{sec:main-closure-theorem}
	
	This section presents the final core conclusion of the paper.
	In previous sections, we have constructed the complete tree model and proven the following key properties:
	
	\begin{itemize}
		\item[(i)] \textbf{Completeness}: The tree contains all positive integers (odd numbers as starting and branch nodes, even numbers as growth nodes).
		
		\item[(ii)] \textbf{Closure}: The branch condition $(m-1)/3 \in \mathbb{N}$ and the growth condition $2m$ assign a unique parent to every node, making the tree internally closed with no external gaps.
		
		\item[(iii)] \textbf{Acyclicity}: The only self-loop occurs in the special degenerate component $4 \rightleftarrows 2 \rightleftarrows 1$; otherwise, the tree is strictly directed and acyclic, with each node (except $1$) strictly decreasing in the potential function.
		
		\item[(iv)] \textbf{Bijection with Collatz Mapping}: The backward paths in the tree correspond one-to-one with the forward Collatz orbits (as proven in the previous section).
	\end{itemize}
	
	Based on these properties, we can prove:
	
	\begin{theorem}[Completeness—Closure—Acyclicity Main Theorem]
		\label{thm:global-convergence}
		In the above tree model, the Collatz iteration of any positive integer $N$,
		\[
		N \mapsto T(N) \mapsto T^2(N) \mapsto \cdots,
		\]
		necessarily reaches the root node $1$ in a finite number of steps. Equivalently,
		\[
		\forall N \in \mathbb{N},\quad \exists k \in \mathbb{N} \text{ such that } T^k(N) = 1.
		\]
	\end{theorem}
	
	\begin{proof}
		By the forward-backward bijection theorem (Theorem~\ref{thm:collatz-bijection}),
		for any $N \in \mathbb{N}$, its Collatz iteration corresponds to a backward path in the tree starting from node $N$.
		
		Hence, it suffices to show that any backward path in the tree reaches the root $1$ in a finite number of steps.
		
		\medskip
		\noindent\textbf{Step 1: Unique parent structure ensures monotonicity.}
		
		Each node in the tree (except $1$) has a unique parent determined by either:
		\[
		m \mapsto \frac{m}{2} \quad (\text{if } m \text{ is even}),\qquad
		m \mapsto \frac{m-1}{3} \quad (\text{if } m \text{ is a branch node}).
		\]
		Thus, backward paths do not branch and can only ascend along a unique chain.
		
		\medskip
		\noindent\textbf{Step 2: Strict decrease of the potential function.}
		
		Define the potential function separately for odd and even layers (as constructed previously):
		\[
		\Phi(n) = (\nu_2(n),\, \mathrm{height}_{\text{odd}}(n)),
		\]
		compared lexicographically.
		
		It has been proven that:
		\begin{itemize}
			\item Backward growth ($n \to n/2$) strictly decreases $\nu_2$;
			\item Backward branching ($n \to (n-1)/3$) keeps $\nu_2 = 0$ but strictly decreases the odd-layer height.
		\end{itemize}
		
		Hence, in every step, the potential function strictly decreases, and no infinite descending chain exists.
		
		\medskip
		\noindent\textbf{Step 3: Acyclicity and termination at the root.}
		
		As proven for both the odd tree and the complete tree, except for the special $4\!\leftrightarrows\!2\!\leftrightarrows\!1$ component, the tree has no directed cycles.
		
		Since the potential function strictly decreases, a backward path cannot revisit a previous node; furthermore, the unique node with minimal potential is $1$, which must be the terminal node of any backward path.
		
		\medskip
		\noindent\textbf{Step 4: Conclusion.}
		
		Therefore, the Collatz orbit of any positive integer $N$ corresponds to a backward path in the tree; this path is finite due to the decreasing potential function, and its unique endpoint is the root node $1$.
		
		Consequently, the Collatz iteration of $N$ converges to $1$ in a finite number of steps.
	\end{proof}
	\subsection{Logical Closure of the Main Theorem}
	
	The main theorem closes the five key structural elements of the entire paper:
	
	\begin{enumerate}
		\item Completeness of the starting set ensures that all odd numbers are included;
		\item Growth rules ensure that all even numbers are present in the tree;
		\item Branch rules guarantee connectivity of all reversible odd numbers in the tree;
		\item The potential function ensures finiteness of backward paths;
		\item Acyclicity guarantees uniqueness of backward paths and a fixed terminal node;
		\item The bijection ensures that convergence in the tree is equivalent to convergence in the Collatz map.
	\end{enumerate}
	
	Hence, the main theorem provides a rigorously closed form of the Collatz conjecture:
	
	\[
	\boxed{
		\text{Global convergence of Collatz iterations to } 1
		\quad\Longleftrightarrow\quad
		\text{Tree model is complete, closed, and acyclic}
	}
	\]
	
	Thus, under this model, the Collatz conjecture is fully resolved.
	
	\section{Discussion of Possible Objections}
	\label{sec:discussion-objections}
	
	This section addresses all potential critical objections to the construction and provides rigorous mathematical responses.
	These discussions ensure that the main theorem is not only internally consistent but also externally verifiable.
	
	\subsection{(1) Question on Uniqueness of Representation}
	
	A possible concern is whether the same node $m$ could appear in different starting points
	$(6n-5),(6n-3),(6n-1)$ with different parameter pairs $(n,x)$ along growth or branch chains, thus producing \emph{multiple representations}.
	
	As rigorously proven earlier:
	
	\begin{itemize}
		\item Growth nodes form strict $2$-adic chains, each with a unique first term (odd starting point), ensuring no overlap;
		\item Branch nodes are generated by the functions
		\[
		B_{1}(n,x) = \frac{(6n-5)2^{2x}-1}{3}, \qquad
		B_{5}(n,x) = \frac{(6n-1)2^{2x-1}-1}{3},
		\]
		which for the same $x$ produce arithmetic sequences with distinct differences and initial terms;
		\item No algebraic intersections exist across different modulo classes;
		\item Therefore, each odd node has a unique origin.
	\end{itemize}
	
	Hence, \textbf{no multi-valued nodes exist in the graph}.  
	If a node had multiple representations, it would satisfy at least two generating functions simultaneously, but the chain of lemmas has already shown that such equations have no solutions.
	
	\subsection{(2) Question on Potential Backward Path Branching}
	
	One might wonder whether a backward path
	\[
	m \to \frac{m}{2} \quad\text{or}\quad m \to \frac{m-1}{3}
	\]
	could simultaneously hold, leading to non-uniqueness.
	
	It is rigorously proven that:
	
	\begin{itemize}
		\item If $m$ is even, $m/2$ is the unique parent;
		\item If $m$ is odd, $m/2\notin \mathbb{N}$, so no growth parent exists;
		\item If $(m-1)/3 \in \mathbb{N}$, it is the unique branch parent;
		\item If $(m-1)/3 \notin \mathbb{N}$, then $m$ cannot be a branch node.
	\end{itemize}
	
	Therefore, \textbf{every node has a unique parent and no backward branching occurs}, consistent with the existence and acyclicity of the tree structure.
	
	\subsection{(3) Question on Possible Hidden Short Cycles}
	
	Another objection might be whether there exist undetected Collatz cycles, e.g., short strange loops.
	
	We have fully enumerated and excluded cycles of length $\le 4$ in a dedicated subsection. Moreover:
	
	\begin{itemize}
		\item The monotonic decrease of the potential function precludes all non-special cycles;
		\item The only structure that can maintain a constant potential is the degenerate $4 \leftrightarrows 2 \leftrightarrows 1$, which is part of the trivial Collatz tail;
		\item Any other cycle length would cause stagnation or increase in the potential function, contradicting the proven lemmas.
	\end{itemize}
	
	Thus, no hidden cycles exist.
	
	\subsection{(4) Question on Whether Adding Even Layers Breaks Acyclicity}
	
	A common concern is whether adding even layers to the odd tree model might create extra cycles.
	
	We have proven that:
	
	\begin{itemize}
		\item Growth chains of even nodes form strict $2$-adic chains with unique parents;
		\item Branch nodes reside in the odd layer, with connections always directed upward;
		\item The only cycle involving even layers is the degenerate $(4,2,1)$ loop;
		\item The potential function $\Phi(n)$ strictly decreases in any backward step (except the above unique loop).
	\end{itemize}
	
	Therefore, adding even layers does not compromise acyclicity.
	\subsection{(5) Question on Potentially Uncovered Integers}
	
	One might ask whether there exist integers that are neither growth nodes nor branch nodes, and thus fail to enter the tree model.
	
	We have already proven that:
	
	\begin{itemize}
		\item All odd numbers are either starting points or branch nodes;
		\item All even numbers are growth nodes of some odd number;
		\item Growth chains
		\[
		(2k) \to (k) \to \cdots \to \text{odd number}
		\]
		always reach an odd number in finitely many steps;
		\item Branch chains cover all odd modulo classes (except $1$) and close with the starting set.
	\end{itemize}
	
	Hence, \textbf{the tree model covers all positive integers, with no missing nodes}.
	
	\subsection{(6) Question on Completeness of Collatz–Tree Correspondence}
	
	Another possible question is whether the forward Collatz orbit truly corresponds one-to-one with the tree’s backward path.
	
	The previous section established the bidirectional correspondence theorem, proving that:
	
	\begin{enumerate}
		\item If $N \to T(N)$ is a forward Collatz mapping, there exists a unique backward edge in the tree model;
		\item If a backward edge exists in the tree model, it corresponds to a unique forward Collatz mapping;
		\item There are no exceptional nodes, including $1$, even chains, or branch odd numbers.
	\end{enumerate}
	
	Thus, \textbf{the two structures form a complete bijection, with no deviations}.
	
	\medskip
	
	In summary, this section addresses all potential structural objections and provides rigorous mathematical support for each.
	Consequently, the construction is both internally logically closed and robust against external critique, giving the main theorem full reliability.
	
	\section{Conclusion}
	\label{sec:conclusion}
	
	Based on the starting set, growth nodes, branch nodes, and connection rules, this paper constructs a directed tree model that contains all positive integers and fully characterizes the dynamics of the Collatz map on $\mathbb{N}$.  
	Through a series of rigorous lemmas, algebraic derivations, potential function analysis, and acyclicity proofs, the core conclusions are as follows:
	
	\begin{enumerate}
		\item \textbf{(Completeness)}  
		The odd part of the tree covers all positive odd numbers.  
		Adding even growth chains yields a complete tree covering all positive integers.  
		Therefore, every positive integer corresponds to a unique node in the tree.
		
		\item \textbf{(Uniqueness of Nodes and Parent Nodes)}  
		From the algebraic structure of the growth and branch rules, each node in the tree has a unique representation and a unique parent, eliminating multiple origins or multi-valued phenomena.
		
		\item \textbf{(Acyclicity and Uniqueness of Special Loops)}  
		Using the strict monotonicity of the potential function $\Phi$, it is proven that no directed cycles exist in the tree except the degenerate self-consistent loop $(4,2,1)$.  
		Therefore, the tree is strictly acyclic over $\mathbb{N}$.
		
		\item \textbf{(Finite Regression of Backward Paths)}  
		Any node can reach the root $1$ in finitely many backward growth or branch steps.  
		This property is ensured by both strict potential function decrease and node uniqueness.
		
		\item \textbf{(Bijection with the Collatz Map)}  
		Backward paths in the tree model are proven to correspond exactly to the forward Collatz process, with no exceptional nodes or mismatched orbits.  
		Each step in the Collatz trajectory is realized as a unique backward edge in the tree.
		
		\item \textbf{(Completeness–Closure Main Theorem)}  
		Under the above structural guarantees, the complete coverage and finite backward regression of the tree imply that any Collatz orbit must reach $1$ in finitely many steps.  
		In other words, Collatz iterations over positive integers have no infinite ascending paths or nontrivial cycles.
	\end{enumerate}
	
	In conclusion, the tree model provides a unified, complete, and rigorous algebraic framework, giving a closed description of all possible behaviors of Collatz iterations.  
	Through systematic analysis of even growth, odd branching, uniqueness, acyclicity, potential function monotonicity, and bidirectional correspondence, the final mathematical conclusion is:
	
	\begin{center}
		\textbf{Every positive integer under the Collatz map eventually reaches $1$ in finitely many steps.}
	\end{center}
	
	This result demonstrates the overall closure and structural origin of the Collatz problem and shows that its dynamics can be fully represented within a rigorously defined tree model.  
	
	Although the construction achieves global control over Collatz iterations, further research directions remain:
	
	\begin{itemize}
		\item Explore whether the potential function structure can be generalized to broader discrete dynamical systems;
		\item Investigate hierarchical distributions, node densities, and statistical properties of the tree model;
		\item Seek possible continuous, analytic, or categorical interpretations;
		\item Apply the tree model to related problems (e.g., Syracuse form, weighted variants, etc.).
	\end{itemize}
	
	These directions can further enrich the understanding of the intrinsic nature of Collatz dynamics.
	
	\appendix
	\section*{Appendix A: Algebraic Exhaustion of Short Cycles (Length $\le 4$)}
	\addcontentsline{toc}{section}{Appendix A: Algebraic Exhaustion of Short Cycles (Length $\le 4$)}
	
	This appendix provides a complete algebraic exclusion of all possible short odd cycles (length $\le 4$) in the odd-number tree model.  
	A short cycle refers to a set of odd nodes $a_1,a_2,\dots,a_k$ ($k\le4$) such that along
	\[
	a_i \xrightarrow{\text{growth/branch}} a_{i+1}, \qquad a_{k+1}=a_1,
	\]
	a closed loop is formed.  
	As already proven in the main text, $1\to4\to2\to1$ is the only special self-loop.  
	This appendix shows that no other odd cycles exist outside this structure.
	
	\subsection*{A.1 \quad Length 1 Cycle (Odd Fixed Point)}
	
	Let $x$ be an odd fixed point, satisfying the odd branch rule:
	\[
	x = \frac{4^m x - 1}{3}, \quad (m\ge 1).
	\]
	Simplifying gives
	\[
	3x = 4^m x - 1 \quad \Longrightarrow \quad (4^m - 3)x = 1.
	\]
	Since $4^m - 3 \ge 1$ and $x$ is a positive integer, the only possibility is
	\[
	4^m - 3 = 1, \quad x = 1,
	\]
	i.e., $m=1$ and $x=1$.  
	Hence, there are no additional odd fixed points.
	
	\subsection*{A.2 \quad Length 2 Cycle (Odd 2-Cycle)}
	
	Let the cycle be $a \mapsto b \mapsto a$, both steps satisfying the odd branch relation:
	\[
	b = \frac{4^{m_1}a - 1}{3}, \qquad a = \frac{4^{m_2}b - 1}{3}.
	\]
	Substituting gives
	\[
	a = \frac{4^{m_2}}{3} \cdot \frac{4^{m_1}a-1}{3} - \frac13
	= \frac{4^{m_1+m_2}a - 4^{m_2} - 3}{9}.
	\]
	Rewriting,
	\[
	9a = 4^{m_1+m_2}a - 4^{m_2} - 3 \quad \Longrightarrow \quad (4^{m_1+m_2}-9)a = 4^{m_2}+3.
	\]
	For $m_1,m_2\ge1$, $4^{m_1+m_2}\ge16$, hence
	\[
	4^{m_1+m_2}-9 \ge 7.
	\]
	The right-hand side $4^{m_2}+3 \equiv 3 \pmod 4$, left-hand coefficient $4^{m_1+m_2}-9 \equiv 3 \pmod4$.  
	Thus, if $a$ exists,
	\[
	a = \frac{4^{m_2}+3}{4^{m_1+m_2}-9}
	\]
	must be a positive odd integer.  
	However, for all $m_1,m_2\ge1$, $4^{m_1+m_2}-9 > 4^{m_2}+3$, so $a<1$, impossible.  
	
	Therefore, no odd 2-cycles exist.
	
	\subsection*{A.3 \quad Length 3 Cycle (Odd 3-Cycle)}
	
	Let
	\[
	a_2 = \frac{4^{m_1} a_1 - 1}{3}, \quad
	a_3 = \frac{4^{m_2} a_2 - 1}{3}, \quad
	a_1 = \frac{4^{m_3} a_3 - 1}{3}.
	\]
	Substituting step by step gives
	\[
	a_1 = \frac{4^{m_3}}{3} \left[ \frac{4^{m_2}}{3} \left( \frac{4^{m_1} a_1 - 1}{3} \right) - \frac13 \right] - \frac13.
	\]
	Expanding,
	\[
	a_1 = \frac{4^{m_1+m_2+m_3}a_1 - 4^{m_2+m_3} - 3\cdot 4^{m_3} - 9}{27}.
	\]
	Rewriting,
	\[
	(4^{m_1+m_2+m_3}-27)a_1 = 4^{m_2+m_3} + 3\cdot 4^{m_3} + 9.
	\]
	The left coefficient
	\[
	4^{m_1+m_2+m_3}-27 \ge 64-27=37,
	\]
	while the right-hand side
	\[
	4^{m_2+m_3} + 3\cdot 4^{m_3} + 9 < 4^{m_1+m_2+m_3}-27.
	\]
	Thus $a_1<1$, impossible.  
	Hence, no odd 3-cycles exist.
	
	\subsection*{A.4 \quad Length 4 Cycle}
	
	Similarly, four-step substitution yields
	\[
	(4^{m_1+m_2+m_3+m_4}-81)a_1 = \text{sum of three terms (clearly } < 4^{m_1+\cdots+m_4}-81\text{)}.
	\]
	Since each $m_i\ge1$, total exponent $\ge4$, hence
	\[
	4^{m_1+\cdots+m_4}-81 \ge 256-81=175.
	\]
	The right-hand side is a sum of terms
	\[
	4^{M_1} + 3\cdot4^{M_2} + 9\cdot4^{M_3} + 27,
	\]
	with largest exponent strictly smaller than $m_1+\cdots+m_4$, so the sum is strictly less than the left coefficient.  
	Thus $a_1<1$, contradiction.
	
	Hence, no odd cycles of length 4 exist.
	
	\subsection*{A.5 \quad Summary: No Short Odd Cycles}
	
	Combining the above four cases:
	
	\begin{itemize}
		\item Length 1 odd cycles only occur at $x=1$;
		\item Length 2, 3, 4 odd cycles do not exist;
		\item Therefore, there are no closed odd loops beyond $1$;
		\item The only closed structure is the special self-loop $1\to4\to2\to1$, already handled in the main text, which does not affect the tree structure.
	\end{itemize}
	
	This completes the algebraic exhaustion and exclusion of all short cycles of length $\le4$.
	\appendix
	\section*{Appendix B: Complete Exhaustion of Cross-Layer Conflicts}
	\addcontentsline{toc}{section}{Appendix B: Complete Exhaustion of Cross-Layer Conflicts}
	\label{app:cross-layer}
	
	This appendix proves that the branch generation functions are uniquely assigned globally according to $(\text{class},x,n)$:  
	there are no overlapping values between different layers ($x\neq x'$) or between different classes ($B_1$ and $B_5$).  
	We first state a general lemma, then exhaustively check the three types of equalities.
	
	\subsection*{Review of Branch Generation Functions and Notation}
	
	Recall the definitions ($x\in\mathbb{Z}_{\ge1}$, $n\in\mathbb{Z}_{\ge1}$):
	\[
	B_1(n,x) = \frac{(6n-5)2^{2x}-1}{3}, \qquad
	B_5(n,x) = \frac{(6n-1)2^{2x-1}-1}{3}.
	\]
	In what follows, whenever we consider an equality $B_\ast(\cdot)=B_\ast(\cdot)$, we multiply both sides by $3$ and remove the common constant $-1$ for algebraic simplicity.  
	This preserves equality while unifying the algebraic form.
	
	\subsection*{Basic Lemma: Uniqueness of Odd × Power of 2}
	
	\begin{lemma}
		\label{lem:odd-times-power-unique}
		Let $u,v$ be odd positive integers, and $A,B\in\mathbb{Z}_{\ge0}$.  
		If
		\[
		u \, 2^A = v \, 2^B,
		\]
		then $A = B$ and $u = v$.
	\end{lemma}
	
	\begin{proof}
		Assume without loss of generality $A\le B$. Then
		\[
		u = v \, 2^{B-A}.
		\]
		If $B-A \ge 1$, the right-hand side is even, but $u$ is odd—a contradiction.  
		Thus $B-A = 0$, i.e., $A = B$, and substituting back gives $u = v$.
	\end{proof}
	
	This lemma is the key tool for excluding conflicts:  
	after multiplying by $3$, all branch generation functions take the form “odd × 2 to a power,”  
	so the lemma immediately provides uniqueness of both the odd factor and the exponent.
	
	\subsection*{Case 1: Same-Class Equality $B_1(n,x) = B_1(n',x')$}
	
	Consider
	\[
	\frac{(6n-5)2^{2x}-1}{3} = \frac{(6n'-5)2^{2x'}-1}{3}.
	\]
	Multiplying both sides by $3$ and removing $-1$ gives
	\[
	(6n-5)2^{2x} = (6n'-5)2^{2x'}.
	\]
	Both factors $(6n-5)$ and $(6n'-5)$ are odd positive integers.  
	By Lemma~\ref{lem:odd-times-power-unique}, we must have
	\[
	2x = 2x' \quad \text{and} \quad 6n-5 = 6n'-5,
	\]
	i.e., $x = x'$ and $n = n'$.  
	Hence, if $(n,x) \neq (n',x')$, there is no solution.  
	\(\mathbf{Different\ layers\ of\ B_1\ sequences\ do\ not\ intersect.}\)
	
	\subsection*{Case 2: Same-Class Equality $B_5(m,x) = B_5(m',x')$}
	
	Similarly, consider
	\[
	\frac{(6m-1)2^{2x-1}-1}{3} = \frac{(6m'-1)2^{2x'-1}-1}{3}.
	\]
	Multiplying by $3$ and removing $-1$ gives
	\[
	(6m-1)2^{2x-1} = (6m'-1)2^{2x'-1}.
	\]
	Both odd factors are odd integers; by Lemma~\ref{lem:odd-times-power-unique}:
	\[
	2x-1 = 2x'-1 \quad \text{and} \quad 6m-1 = 6m'-1,
	\]
	i.e., $x = x'$ and $m = m'$.  
	Hence, \(\mathbf{Different\ layers\ of\ B_5\ sequences\ do\ not\ intersect.}\)
	
	\subsection*{Case 3: Cross-Class Equality $B_1(n,x) = B_5(m,x')$}
	
	Consider the possible cross-class equality:
	\[
	\frac{(6n-5)2^{2x}-1}{3} = \frac{(6m-1)2^{2x'-1}-1}{3}.
	\]
	Multiplying by $3$ and removing $-1$ gives
	\[
	(6n-5)2^{2x} = (6m-1)2^{2x'-1}. \tag{*}
	\]
	Observing the powers of $2$: left side is $2^{2x}$ (even exponent), right side is $2^{2x'-1}$ (odd exponent).  
	Thus $2x = 2x'-1$, impossible in integers.  
	Hence (*) has no solution for any integers $x,x'$.  
	\(\mathbf{No\ B_1\ term\ can\ equal\ any\ B_5\ term.}\)
	
	\subsection*{General Remarks: Exponent Parity and Odd Factor Consistency}
	
	The key in all three cases is:
	
	\begin{enumerate}
		\item After multiplying by $3$, both sides are in the form “odd × $2^{\text{exponent}}$,” so Lemma~\ref{lem:odd-times-power-unique} applies directly.
		\item Exponents of $B_1$ are always even ($2x$), while exponents of $B_5$ are always odd ($2x-1$).  
		Their parity prevents any cross-class equality.
	\end{enumerate}
	
	Thus cross-layer and cross-class conflicts are trivially excluded.
	
	\subsection*{Conclusion Theorem}
	
	\begin{theorem}[No Cross-Layer Conflicts]
		For any positive integer parameters $(n,n',m,m')$ and layer parameters $x,x'\in\mathbb{Z}_{\ge1}$, the following equalities hold only in the trivial identical parameter cases:
		\[
		B_1(n,x) = B_1(n',x'), \qquad
		B_5(m,x) = B_5(m',x'), \qquad
		B_1(n,x) = B_5(m,x').
		\]
		More specifically:
		\begin{enumerate}[label=(\arabic*)]
			\item $B_1(n,x) = B_1(n',x') \iff x=x'\ \text{and}\ n=n'$;
			\item $B_5(m,x) = B_5(m',x') \iff x=x'\ \text{and}\ m=m'$;
			\item For any parameters, $B_1(n,x) = B_5(m,x')$ is impossible.
		\end{enumerate}
		That is, \emph{branch generation sequences are globally uniquely assigned according to $(\text{class},x,n)$, with no cross-layer or cross-class conflicts.}
	\end{theorem}
	
	\begin{proof}
		All three cases have been proven above; the theorem follows immediately.
	\end{proof}
	
	\subsection*{Supplementary Remark (Range of $x$ and Boundary Cases)}
	
	In practice, we usually take $x\ge1$ corresponding to the actual branch-generating $k$ ($k=2x$ or $k=2x-1$).  
	The lemma and theorem arguments still hold for wider ranges of $x$ (e.g., $x\ge0$),  
	with only minor attention needed to ensure integers exist for $x=0$.  
	The algebraic structure and parity arguments are robust at the boundary,  
	so the conclusion is generally valid.
	
	\qed
	\section*{Appendix C: Formal Definition of the Graph Structure}
	\addcontentsline{toc}{section}{Appendix C: Formal Definition of the Graph Structure}
	\label{app:graph}
	
	This appendix provides a formal graph-theoretic definition of the tree models (the odd tree and the complete tree). The purpose is to abstract all structures in the main text, such as “nodes,” “growth edges,” “branch edges,” and “connection edges,” into a unified directed graph, allowing all subsequent theorems to be formulated and verified within standard mathematical structures.
	
	\subsection*{C.1 Basic Notation}
	
	Let
	\[
	\mathbb{N}_{\mathrm{odd}} = \{1,3,5,\dots\},\qquad
	\mathbb{N}_{\mathrm{even}} = \{2,4,6,\dots\},\qquad
	\mathbb{N}_{>0} = \{1,2,3,\dots\}.
	\]
	
	In the main text, the odd tree contains all positive odd integers, whereas the complete tree contains all positive integers.
	
	\subsection*{C.2 Definition of the Graph Structure}
	
	\begin{definition}[Directed Tree Graph Structure]
		\label{def:directed-tree-structure}
		We define a tree model as a quadruple
		\[
		\mathcal{T} = (V,E,r,\pi),
		\]
		where:
		
		\begin{enumerate}[label=(\arabic*)]
			\item $V$ is the set of nodes, a subset of $\mathbb{N}_{>0}$. For the odd tree, $V=\mathbb{N}_{\mathrm{odd}}$; for the complete tree, $V=\mathbb{N}_{>0}$.
			
			\item $E \subseteq V\times V$ is the set of edges, each directed, denoted $u \to v$. The edge direction is determined by the construction rules (growth, branching, connection) and always points toward the “newly generated node.”
			
			\item $r \in V$ is the root node. In this tree, $r=1$.
			
			\item $\pi : V\setminus\{r\} \to V$ is the \emph{parent map}, satisfying
			\[
			v\to \pi(v)\in E, \qquad \forall\, v\neq r.
			\]
			That is, each non-root node has exactly one parent.
		\end{enumerate}
		
		Moreover, $(V,E)$ contains no directed cycles except for possible special self-loops around the root.
	\end{definition}
	
	The root node $1$ has no parent, while all other nodes have a unique parent; thus $\mathcal{T}$ is a directed tree rooted at $1$.
	
	\subsection*{C.3 Formal Definitions of Growth, Branch, and Connection Edges}
	
	Edges in the complete tree are defined by three generation rules:
	
	\begin{definition}[Growth Edge]
		\label{def:growth-edge}
		For $u\in \mathbb{N}_{\mathrm{odd}}$ and $v=u\cdot 2^k$ generated by the growth rule ($k\ge1$), define
		\[
		u\to v \in E.
		\]
		This produces the chain from odd nodes to even nodes.
	\end{definition}
	
	\begin{definition}[Branch Edge (Branch Node)]
		\label{def:branch-edge}
		If a node $v$ (even only) satisfies $(v-1)/3\in\mathbb{N}_{>0}$ and the quotient is odd, generate a branch node
		\[
		u=\frac{v-1}{3},
		\]
		and define the edge
		\[
		v\to u \in E.
		\]
	\end{definition}
	
	\begin{definition}[Connection Edge (Merging Across Classes)]
		\label{def:connect-edge}
		If a branch node $u$ has the same value as a starting node $u'$, identify $u$ and $u'$ as the same node and merge their respective subtrees. No new edge is added; only the consistency and uniqueness of the parent map $\pi$ are preserved.
	\end{definition}
	
	Connection edges essentially do not introduce new edges; they merge node identities so that different generation sources eventually converge to the same node.
	
	Edges in the odd tree are similarly defined by three generation rules:
	
	\begin{definition}[Growth Edge]
		For each odd starting node $s$ and all its branch nodes $b\in B(s)$, define a growth edge from $s$ to $b$:
		\[
		(s,b)\in E_g.
		\]
		Equivalently:
		\[
		E_g = \{(s,b) \mid s \in V,\; b \in B(s)\}.
		\]
	\end{definition}
	
	\begin{definition}[Branch Edge / Connection Edge]
		\label{def:branch-edge}
		If there exists an odd $s\ne b$ such that
		\[
		b\in B(s) \quad \Longleftrightarrow \quad (s,b)\in E_g,
		\]
		then this edge is treated as a \textbf{branch edge} or \textbf{connection edge} in the odd tree:
		\[
		(s,b)\in E_b.
		\]
		
		Its meaning is that structures generated from different odd starting points merge at the odd node $b$.
		
		The odd tree and the complete tree are essentially the same; the only difference is the node set. The odd tree is formed by removing even nodes from the complete tree.
	\end{definition}
	
	\subsection*{C.4 Reverse Edges and the Reverse Graph}
	
	To describe the “reverse regression” process (corresponding to the Collatz descent), we define reverse edges:
	
	\begin{definition}[Reverse Edge and Reverse Graph]
		Define the set of reverse edges
		\[
		E^{-1} = \{\, (v,u) \in V\times V : u \to v \in E\,\}.
		\]
		The reverse graph is
		\[
		\mathcal{T}^{-1} = (V,E^{-1}).
		\]
		Its edges represent the reverse regression relation: if $v$ points to $u$ in the reverse graph, then $u$ is the parent $\pi(v)$ of $v$.
	\end{definition}
	
	In particular:
	- The root node $1$ has no reverse edge.
	
	\subsection*{C.5 Depth and Layer}
	
	\begin{definition}[Depth]
		Define a depth function $\mathrm{depth}: V \to \mathbb{Z}_{\ge 0}$ as the minimum number of steps to reach a node from the root $1$ via the parent map:
		\[
		\mathrm{depth}(1)=0,\qquad
		\mathrm{depth}(v) = 1 + \mathrm{depth}(\pi(v)).
		\]
	\end{definition}
	
	\begin{definition}[Layer]
		The $k$-th layer is defined as
		\[
		L_k = \{v \in V : \mathrm{depth}(v) = k\}.
		\]
	\end{definition}
	
	Thus, the graph structure is fully formalized as a directed acyclic tree.
	
	\subsection*{C.6 Correctness Requirements of the Graph Structure}
	
	The construction rules must satisfy the following consistency conditions:
	
	\begin{enumerate}[label=(\arabic*)]
		\item Every $v\neq 1$ must have a unique parent $\pi(v)$;
		\item No non-trivial directed cycles $v\to \cdots \to v$ exist;
		\item Node identity merging (connection rule) must not produce multiple parents;
		\item In the complete tree model: the odd subtree is embedded as an induced subgraph; even nodes do not violate tree properties.
	\end{enumerate}
	
	This section ensures the mathematical clarity and verifiability of the tree model and provides a unified foundation for all theorems in the main text.
	
	\bibliographystyle{plain}
	\bibliography{references}
	
	
\end{document}
	